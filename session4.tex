\documentclass[11pt]{article}

\usepackage{quiver}
\usepackage{amsmath, amsthm, mathrsfs}
\usepackage[margin = 3cm]{geometry}
\usepackage{enumitem}
\usepackage{csquotes}
\usepackage{stmaryrd}

\newtheorem{theorem}{Theorem}[section]
\newtheorem{proposition}[theorem]{Proposition}
\newtheorem{lemma}[theorem]{Lemma}
\newtheorem{corollary}[theorem]{Corollary}
\theoremstyle{definition}
\newtheorem{definition}[theorem]{Definition}
\newtheorem{remark}[theorem]{Remark}
\newtheorem{example}[theorem]{Example}
\newtheorem{exercise}[theorem]{Exercise}

\newcommand{\A}{\mathbf{A}}
\newcommand{\aff}{\mathrm{aff}}
\newcommand{\Ani}{\operatorname{Ani}}
\newcommand{\ani}{\mathrm{ani}}
\newcommand{\aug}{\mathrm{aug}}
\newcommand{\C}{\mathcal{C}}
\newcommand{\CAlg}{\mathrm{CAlg}}
\newcommand{\Cart}{\operatorname{Cart}}
\newcommand{\Cat}{\mathrm{Cat}}
\newcommand{\Catinf}{\mathcal{C}\mathrm{at}_{\infty}}
\newcommand{\cc}{\mathrm{cc}}
\newcommand{\cl}{\mathrm{cl}}
\newcommand{\Cof}{\mathrm{Cof}}
\newcommand{\colim}{\operatorname*{colim}}
\newcommand{\CRMod}{\mathrm{CRMod}}
\newcommand{\Cut}{\operatorname{Cut}}
\newcommand{\D}{\mathcal{D}}
\newcommand{\Der}{\operatorname{Der}}
\newcommand{\dSch}{\mathrm{dSch}}
\newcommand{\E}{\mathcal{E}}
\newcommand{\Einf}{\mathbb{E}_{\infty}}
\newcommand{\Eq}{\operatorname{Eq}}
\newcommand{\et}{\mathrm{\acute{e}t}}
\newcommand{\FFree}{\mathrm{FFree}}
\newcommand{\Fib}{\mathrm{Fib}}
\newcommand{\for}{\mathrm{for}}
\newcommand{\Fun}{\operatorname{Fun}}
\newcommand{\Grpd}{\mathrm{Grpd}}
\newcommand{\h}{\mathrm{h}}
\newcommand{\heart}{\heartsuit}
\newcommand{\Hom}{\operatorname{Hom}}
\newcommand{\iHom}{\operatorname{\underline{Hom}}}
\newcommand{\Ind}{\operatorname{Ind}}
\newcommand{\J}{\mathcal{J}}
\newcommand{\Kan}{\mathrm{Kan}}
\renewcommand{\L}{\mathbb{L}}
\newcommand{\LSym}{\L\mathrm{Sym}}
\newcommand{\M}{\mathscr{M}}
\newcommand{\Map}{\operatorname{Map}}
\newcommand{\Mod}{\mathrm{Mod}}
\newcommand{\N}{\mathrm{N}}
\newcommand{\Ob}{\operatorname{Ob}}
\newcommand{\op}{\mathrm{op}}
\newcommand{\p}{\mathfrak{p}}
\renewcommand{\P}{\mathcal{P}}
\newcommand{\Perf}{\mathrm{Perf}}
\newcommand{\Poly}{\mathrm{Poly}}
\newcommand{\PreStack}{\mathrm{PreStack}}
\newcommand{\PrL}{\mathrm{Pr}^{\mathrm{L}}}
\newcommand{\PrR}{\mathrm{Pr}^{\mathrm{R}}}
\newcommand{\pr}{\mathrm{pr}}
\newcommand{\proj}{\mathrm{proj}}
\newcommand{\Q}{\mathbf{Q}}
\newcommand{\QCoh}{\operatorname{QCoh}}
\newcommand{\SCR}{\mathrm{SCR}}
\newcommand{\Set}{\mathrm{Set}}
\newcommand{\Sp}{\mathcal{S}\mathrm{p}}
\newcommand{\Spc}{\mathcal{S}\mathrm{pc}}
\newcommand{\Spec}{\operatorname{Spec}}
\newcommand{\sSet}{\mathrm{sSet}}
\newcommand{\Stack}{\mathrm{Stack}}
\newcommand{\Tor}{\operatorname{Tor}}
\newcommand{\X}{\mathcal{X}}
\newcommand{\Y}{\mathcal{Y}}
\newcommand{\Z}{\mathbf{Z}}


\title{Lecture notes in Derived Algebraic Geometry\\Session 4}
\date{21 February 2025}
\author{Course by F. Binda, notes by E. Hecky}


\begin{document}
\maketitle

The aim of this last session is to finish the construction of the cotangent complex, and to discuss a bit about how to globalize the ring-theoretic construction that were introduced in the previous session.

\section{Square-zero extensions and the cotangent complex}

This section is actually the continuation of the last section of the previous lesson.
Recall that we defined an augmented $A$-algebra $A \oplus M$ for every animated module $M$ over an animated ring $A$, satisfying intuitive basic properties.

\begin{lemma}
    There exists a fiber sequence
    \[
        M \to A \oplus M \to A
    \]
    where the second map is an effective epimorphism, and a commutative diagram of $\infty$-categories
    % https://q.uiver.app/#q=WzAsOCxbMCwwLCJcXEFuaShcXENSTW9kXntcXGhlYXJ0fSkiXSxbMSwwLCJcXEFuaShcXENSTW9kXntcXGhlYXJ0fSkiXSxbMCwxLCJcXENBbGdee1xcYW5pfSJdLFsxLDEsIlxcQ0FsZ157XFxhbml9Il0sWzIsMCwiKEEsIE0pIl0sWzMsMCwiKEEsIDApIl0sWzIsMSwiQSBcXG9wbHVzIE0iXSxbMywxLCJBIl0sWzAsMV0sWzAsMiwiXFxvcGx1cyIsMl0sWzIsM10sWzEsMywiXFxwcl8xIl0sWzQsNiwiIiwwLHsic3R5bGUiOnsidGFpbCI6eyJuYW1lIjoibWFwcyB0byJ9fX1dLFs0LDUsIiIsMix7InN0eWxlIjp7InRhaWwiOnsibmFtZSI6Im1hcHMgdG8ifX19XSxbNSw3LCIiLDIseyJzdHlsZSI6eyJ0YWlsIjp7Im5hbWUiOiJtYXBzIHRvIn19fV0sWzYsNywiIiwwLHsic3R5bGUiOnsidGFpbCI6eyJuYW1lIjoibWFwcyB0byJ9fX1dXQ==
    \[\begin{tikzcd}
        {\Ani(\CRMod^{\heart})} & {\Ani(\CRMod^{\heart})} & {(A, M)} & {(A, 0)} \\
        {\CAlg^{\ani}} & {\CAlg^{\ani}} & {A \oplus M} & A
        \arrow[from=1-1, to=1-2]
        \arrow["\oplus"', from=1-1, to=2-1]
        \arrow["{\pr_1}", from=1-2, to=2-2]
        \arrow[maps to, from=1-3, to=1-4]
        \arrow[maps to, from=1-3, to=2-3]
        \arrow[maps to, from=1-4, to=2-4]
        \arrow[from=2-1, to=2-2]
        \arrow[maps to, from=2-3, to=2-4]
    \end{tikzcd}\]
\end{lemma}

Remember that classically, $\Der(A, M) \simeq \Hom_{/A}(A, A \oplus M)$.
This motivates the definition of derivations in the animated context.

\begin{definition}
    Let $(A, M) \in \Ani(\CRMod^{\heart})$.
    Then define
    \[
        \Der(A, M) = \Map_{\CAlg_{/A}^{\ani}}(A, A \oplus M).
    \]
\end{definition}

\begin{remark}
    $A \oplus M$ is a loop space !
    Indeed, there is a pullback square of the form
    % https://q.uiver.app/#q=WzAsNCxbMCwwLCJBIFxcb3BsdXMgTSJdLFsxLDAsIkEiXSxbMCwxLCJBIl0sWzEsMSwiQSBcXG9wbHVzIE1bMV0iXSxbMCwxXSxbMCwyXSxbMiwzXSxbMSwzXSxbMCwzLCIiLDEseyJzdHlsZSI6eyJuYW1lIjoiY29ybmVyIn19XV0=
    \[\begin{tikzcd}
        {A \oplus M} & A \\
        A & {A \oplus M[1]}
        \arrow[from=1-1, to=1-2]
        \arrow[from=1-1, to=2-1]
        \arrow["\lrcorner"{anchor=center, pos=0.125}, draw=none, from=1-1, to=2-2]
        \arrow[from=1-2, to=2-2]
        \arrow[from=2-1, to=2-2]
    \end{tikzcd}\]

    Also, $\Der(A, M)$ is the fiber of the map
    \[
        \Map_{\CAlg^{\ani}}(A, A \oplus M) \to \Map_{\CAlg^{\ani}}(A, A).
    \]
\end{remark}

\begin{proposition}
    The functor
    \[
        \Der(A, -) : \Mod_A \to \Spc
    \]
    is accessible and preserves limits.
    Therefore, it has a left adjoint
    \[
        G : \Spc \to \Mod_A.
    \]
\end{proposition}
\begin{proof}
    For the accessibility, it is enough to observe that $\Map(A, -)$ is accessible since $M \mapsto A \oplus M$ commutes with sifted colimits by design (it is an animation), so it is accessible.

    To show that the functor preserves limits, notice that the forgetful functor $\CAlg_{/A}^{\ani} \to \CAlg^{\ani}$ detects limits, so it is enough to show that if $p : K \to \Mod_A$ is a diagram of $A$-modules, then
    \[
        \Map_{/A}(A[\underline{T}], A \oplus \lim_K p) \simeq \lim_K \Map_{/A}(A[\underline{T}], A \oplus p).
    \]
    Why is this the case ?
    Recall that there is an adjunction :
    \[
        \LSym_{\Z} : \Mod_{\Z}^{\ani} \rightleftarrows \CAlg^{\ani} : \for
    \]
    between the (derived) free symmetric algebra and forgetful functors.
    Apply this adjunction and recall that small limits in $\Mod_A$ commute with $\oplus$ (which is simply the product in $\Mod_A$) to obtain :
    \begin{align*}
        \Map_{\CAlg_{/A}}(A[\underline{T}], A \oplus \lim_K p) &\simeq \Map_{\Mod_A}(A^{\oplus n}, \for(A \oplus \lim_K p))\\
        &\simeq \lim_K \Map_{\Mod_A}(A^{\oplus n}, \for(A \oplus p))\\
        &\simeq \lim_K \Map_{\CAlg_{/A}}(\LSym(A^{\oplus n}) = A[\underline{T}], A \oplus p).
    \end{align*}
\end{proof}

One can now define the cotangent complex.
\begin{definition}
    The \emph{absolute cotangent complex} $\L_A$ is the module $G(\ast)$.
\end{definition}

\begin{remark}
    There is another, spectral way to construct the cotangent complex.
    If $\C$ is a symmetric monoidal stable $\infty$-category, then one can define a functor $G : \CAlg^{\aug}(\C) \to \C$ sending an augmented algebra $f : A \to \mathbf{1}$ to its cofiber.
    There is a theorem stating that $G$\footnote{or more precisely, its Goodwillie derivative} induces an equivalence
    \[
        \partial G : \Sp(\CAlg^{\aug}(\C)) \simeq \Sp(\C) \simeq \C
    \]
    which can be used to define the cotangent complex.
    The advantage of this approach is that it lets us defined $A \oplus M$ for any non-necessarily animated $\Einf$-ring spectrum $A$.
\end{remark}

\begin{theorem}
    The construction $A \mapsto \L_A$ extends to a functor
    \[
        \CAlg^{\ani} \to \Ani(\CRMod^{\heart})
    \]
    sending $f : A \to B$ to $B \otimes_A^{\L} \L_A \to \L_B$.
    If $A$ is a polynomial algebra then $\L_A \simeq \Omega_{A/\Z}$.
\end{theorem}

\begin{definition}
    For an animated ring map $f : A \to B$, its \emph{relative cotangent complex} $\L_{B/A}$ is the cofiber of the map $\L_A \otimes_A^{\L} B \to \L_B$.
\end{definition}

\begin{proposition}
    There is an equivalence
    \[
        \Map_{\Mod_B}(\L_{B/A}, N) \simeq \Map_{\CAlg_{A//B}}(B, B \oplus N).
    \]
\end{proposition}

Define the space $\Der_A(B, N)$ of $A$-linear derivations from $B$ to $N$ to be this space.

\begin{proposition}
    The cotangent complex satisfies the following basic properties :
    \begin{enumerate}
        \item (Base change) If $B \simeq B' \otimes_{A'} A$, then $\L_{B/A} \simeq \L_{B'/A'} \otimes B$ ;
        \item If $A \to B \to C$ is a fiber sequence, then there is a fiber sequence
        \[
            C \otimes_B \L_{B/A} \to \L_{C/A} \to \L_{C/B}\text{ ;}
        \]
        \item For every map $f : A \to B$, there is an associated map $\varepsilon(f) : B \otimes_A \Cof(f) \to \L_{B/A}$.
        We have the following connectivity estimate : if the fiber of $f$ is connective, then the fiber of $\varepsilon(f)$ is 2-connective.
    \end{enumerate}
\end{proposition}

An important \enquote{slogan} in derived algebraic geometry which gives even more importance to this object is the idea that all animated rings can be built from discrete rings with square-zero extensions.

\begin{definition}
    A map $S' \to S$ in $\CAlg^{\ani}$ is a \emph{square-zero extension} if there is an $S$-module $M$ and a derivation $d \in \pi_0 \Map_{/S}(S, S\oplus M[1])$ such that the following square is cartesian :
    % https://q.uiver.app/#q=WzAsNCxbMCwwLCJTJyJdLFsxLDAsIlMiXSxbMCwxLCJTIl0sWzEsMSwiUyBcXG9wbHVzIE1bMV0iXSxbMCwxXSxbMCwyXSxbMSwzLCJkXzAiXSxbMiwzLCJkIiwyXV0=
    \[\begin{tikzcd}
        {S'} & S \\
        S & {S \oplus M[1]}
        \arrow[from=1-1, to=1-2]
        \arrow[from=1-1, to=2-1]
        \arrow["{d_0}", from=1-2, to=2-2]
        \arrow["d"', from=2-1, to=2-2]
    \end{tikzcd}\]
    Here, $d_0$ denotes the trivial derivation $S \to S \oplus M[1]$ which is just the inclusion of the first factor.
\end{definition}

\begin{proposition}[Lurie, DAG]
    Let $M$ be an animated module over $S$.
    If $S'$ is a square-zero extension of $S$ by $M$ and if $R \to S$ is any ring map, then the space of lifts $\Map_{\CAlg_{/S}^{\ani}}(R, S')$ is a torsor under $\Der(R, M)$.
\end{proposition}

Concretely, this means that to get a lift $R \to S'$ in the following diagram
% https://q.uiver.app/#q=WzAsNSxbMCwwLCJSIl0sWzEsMCwiUyciXSxbMiwwLCJTIl0sWzEsMSwiUyJdLFsyLDEsIlMgXFxvcGx1cyBNWzFdIl0sWzEsMl0sWzEsM10sWzAsM10sWzAsMSwiIiwwLHsic3R5bGUiOnsiYm9keSI6eyJuYW1lIjoiZGFzaGVkIn19fV0sWzMsNCwiZCIsMl0sWzIsNCwiZF8wIl0sWzEsNCwiIiwyLHsic3R5bGUiOnsibmFtZSI6ImNvcm5lciJ9fV1d
\[\begin{tikzcd}
	R & {S'} & S \\
	& S & {S \oplus M[1]}
	\arrow[dashed, from=1-1, to=1-2]
	\arrow[from=1-1, to=2-2]
	\arrow[from=1-2, to=1-3]
	\arrow[from=1-2, to=2-2]
	\arrow["\lrcorner"{anchor=center, pos=0.125}, draw=none, from=1-2, to=2-3]
	\arrow["{d_0}", from=1-3, to=2-3]
	\arrow["d"', from=2-2, to=2-3]
\end{tikzcd}\]
it is sufficient and necessary that the derivation $R \to S \xrightarrow{d} S \oplus M[1]$ in $\pi_0 \Der(R, M[1]) = \pi_0 \Map_{\Mod_R}(\L_R, M[1])$ is the trivial derivation.

\begin{example}
    Postnikov towers give rise to square zero extensions !
    Recall that for every $n \geq 0$, there is a functor
    \[
        \tau_{\leqslant n} : \CAlg^{\ani} \to \CAlg^{\ani}.
    \]
    Then $\tau_{\leqslant n} R$ is a square-zero extension of $\tau_{\leqslant n-1} R$ by $(\pi_n R)[n]$, i.e. there is a pullback square :
    % https://q.uiver.app/#q=WzAsNCxbMCwwLCJcXHRhdV97XFxsZXFzbGFudCBufVIiXSxbMSwwLCJcXHRhdV97XFxsZXFzbGFudCBuLTF9UiJdLFswLDEsIlxcdGF1X3tcXGxlcXNsYW50IG4tMX1SIl0sWzEsMSwiXFx0YXVfe1xcbGVxc2xhbnQgbi0xfVIgXFxvcGx1cyAoXFxwaV9uUilbbisxXSJdLFsxLDMsImRfMCJdLFsyLDMsIlxcZXhpc3RzIGQiLDIseyJzdHlsZSI6eyJib2R5Ijp7Im5hbWUiOiJkYXNoZWQifX19XSxbMCwyXSxbMCwxXSxbMCwzLCIiLDEseyJzdHlsZSI6eyJuYW1lIjoiY29ybmVyIn19XV0=
    \[\begin{tikzcd}
        {\tau_{\leqslant n}R} & {\tau_{\leqslant n-1}R} \\
        {\tau_{\leqslant n-1}R} & {\tau_{\leqslant n-1}R \oplus (\pi_nR)[n+1]}
        \arrow[from=1-1, to=1-2]
        \arrow[from=1-1, to=2-1]
        \arrow["\lrcorner"{anchor=center, pos=0.125}, draw=none, from=1-1, to=2-2]
        \arrow["{d_0}", from=1-2, to=2-2]
        \arrow["{\exists d}"', dashed, from=2-1, to=2-2]
    \end{tikzcd}\]
    This explains the philosophy of derived algebraic geometry, that after passing from varieties to schemes by adding non-reduced points, we pass from schemes to derived schemes by adding finer infinitesimal information.
    This Postnikov construction allows for handy inductive arguments using the cotangent complex.
\end{example}

\paragraph{Sketch of why this is true}
One easy thing to see first is that $\tau_{\leqslant n}R$ should equalize the two maps to the tensor product :
\[
    \tau_{\leqslant n} R \to \tau_{\leqslant n-1}R \rightrightarrows \tau_{\leqslant n-1}R \otimes_{\tau_{\leqslant n} R} \tau_{\leqslant n-1}R
\]
(even though this might not be an equalizer diagram).
Apply the $\tau_{\leqslant n+1}$ functor to this to get
\[
    \tau_{\leqslant n} R \to \tau_{\leqslant n-1}R \rightrightarrows \underbrace{\tau_{\leqslant n+1}(\tau_{\leqslant n-1}R \otimes_{\tau_{\leqslant n} R} \tau_{\leqslant n-1}R)}_{\star}.
\]
What is $\star$ ?
We can compute its homotopy groups with the long exact sequence and check that $\pi_*(\star) \simeq \pi_*(\tau_{\leqslant n-1}R \oplus \pi_n(R)[n+1])$.
This should be a good clue that $\star$ is homotopy equivalent to the bottom-right object of the square in the statement.
This would in turn show that there is a map from $\tau_{\leqslant n}R$ to the pullback.
Of course, this all depends on the existence of a suitable derivation $d$, which will not be discussed.

\section{Another word on geometrical aspects}

\begin{definition}
    Let $k \to R$ be a map of animated rings.
    We say that $R$ is \emph{derived formally smooth over $k$} if for every square-zero extension of $k$-algebras $S' \to S$ and for every map $R \to S$, there exists a lift in the following diagram of $k$-algebras :
    % https://q.uiver.app/#q=WzAsMyxbMCwwLCJSIl0sWzEsMCwiUyJdLFsxLDEsIlMnIl0sWzIsMV0sWzAsMV0sWzAsMiwiIiwwLHsic3R5bGUiOnsiYm9keSI6eyJuYW1lIjoiZGFzaGVkIn19fV1d
    \[\begin{tikzcd}
        R & S \\
        & {S'}
        \arrow[from=1-1, to=1-2]
        \arrow[dashed, from=1-1, to=2-2]
        \arrow[from=2-2, to=1-2]
    \end{tikzcd}\]
\end{definition}

\begin{proposition}
    With the same notation, being derived formally smooth is equivalent to the vanishing of $\pi_0 \Map(\L_{R/k}, M[1])$ for every $M \in \Mod_R$.
\end{proposition}

\begin{definition}
    With the same notation as above and in the derived formally smooth assumption, we say that $R$ is \emph{derived formally étale} if moreover the space of lifts is contractible.
\end{definition}

\begin{proposition}
    Being derived formally étale is equivalent to being formally étale in the sense of the definition from the previous session (namely, that the $\pi_0$ is étale and that $\pi_i(R) \simeq \pi_i(k) \otimes_{\pi_0(k)} \pi_0(R))$.
\end{proposition}

\begin{exercise}
    Let $R$ be an animated ring and $x_1, \dots, x_c \in \pi_0 R$.
    Then prove that
    \[
        \L_{\left(R\sslash(x_1, \dots, x_c)\right)/R} \simeq R\sslash(x_1, \dots, x_c)^{\oplus c}[1].
    \]
    If $\mathcal{J}$ is the fiber of the map $R \to R\sslash(x_1, \dots, x_c)$, then
    \[
        \L \simeq \mathcal{J} \otimes_R R\sslash(x_1, \dots, x_c)[1].
    \]
    Concretely, the cotangent complex \enquote{contains} the conormal bundle.
    To be more precise, if $A \to B$ is a map of static rings in $\CAlg^{\heart}$, then
    \[
        \tau_{\leqslant 1} \L_{B/A} \simeq \left[\mathcal{I}/\mathcal{I}^2 \to \Omega_A \otimes_A B\right].
    \]
\end{exercise}

\begin{theorem}[Quillen, Lurie]
    Let $f : A \to B$ be a map whose cofiber is $n$-connective.
    Then $\varepsilon(f)$ is $(n+2)$-connective.
    Moreover, $f$ is an equivalence is and only if $\L_{B/A} \simeq 0$ and $\pi_0(A) \simeq \pi_0(B)$.
\end{theorem}

\begin{definition}
    Let $R$ be an animated ring and $M \in \Mod_R$.
    We say that $M$ has \emph{Tor-amplitude in $[a, b]$} if for every static module $N \in \Mod_{\pi_0R}^{\heart}$, we have $H_i(M \otimes_R^{\L} N) = 0$ whenever $i \notin [a, b]$.
\end{definition}

The following theorem, conjectured by Quillen decades earlier, is striking : it explains that under mild assumptions, the cotangent complex is either extremely simple (concentrated in degrees 0 and 1) or extremely complex (unbounded).

\begin{theorem}[Avramov]
    If $f : k \to R$ is a map of noetherian rings, then it is locally complete intersection if and only if $R$ is locally of finite flat dimension and $\L_{R/k}$ has bounded Tor-amplitude.
\end{theorem}

\begin{proposition}
    Let $A$ be an animated ring and $B \in \CAlg_A^{\omega}$ be a compact $A$-algebra.
    Then $\L_{B/A}$ is also compact as a $B$-module.
    The converse is also true if $\pi_0(B)$ is of finite presentation over $\pi_0(A)$.
\end{proposition}

\section{Global derived geometry}

This last section aims to give a glimpse on how one can globalize the ring-theoretic constructions that have been discussed up until now, in order to define derived schemes.

\subsection{Prestacks and affine derived schemes}

There are two approaches, the first using ringed topoi and looking like the \enquote{locally ringed space} approach to schemes.
The second one, which will be followed here, is the \enquote{functor of points and descent conditions} approach.
It is in a sense more natural, at least algebraically.

\begin{definition}
    Let $\PreStack = \Fun(\CAlg^{\ani}, \Spc)$ be the category of functors on derived rings.
    By Yoneda, any animated ring $A$ defines a representable prestack $\Spec(A) = \Map_{\CAlg^{\ani}}(A, -)$.
    Let $\dSch^{\aff}$ be the full subcategory of $\PreStack$ spanned by these affine derived schemes $\Spec(A)$.
\end{definition}

Recall that there is an adjunction between the inclusion $\CAlg^{\heart} \hookrightarrow \CAlg^{\ani}$ and the $\pi_0$ functor : this yields an adjunction by further abstract nonsense :
\[
    \pi_0^* : \PreStack^{\cl} \rightleftarrows \PreStack : (-)^{\cl}
\]
where $\PreStack^{\cl} = \Fun(\CAlg^{\heart}, \Spc)$.

\begin{remark}
    For every animated ring $A$, we have $\Spec(A)^{\cl} = \Spec(\pi_0 A)$.
\end{remark}

\subsection{Grothendieck topologies}

Let us define the Zariski and étale topologies on $\dSch^{\aff}$.

\begin{definition}
    Let $I$ be a set and $\{j_{\alpha} : \Spec(B_{\alpha}) \to \Spec(A)\}_{\alpha \in I}$ be a family of maps in $\dSch^{\aff}$.
    Such a family is a \emph{Zariski covering} (resp. \emph{étale covering}) if the following conditions hold.
    \begin{enumerate}
        \item The functor $\Mod_A \to \prod_{\alpha} \Mod_{B_{\alpha}}$ is conservative ;
        \item Each map $A \to B_{\alpha}$ is Zariski, i.e. it is an open immersion at the $\pi_0$ level and $\pi_i(B_{\alpha}) \simeq \pi_i(A) \otimes_{\pi_0(A)} \pi_0(B_{\alpha})$ (resp. each map is (formally) étale in the sense already defined earlier).
    \end{enumerate}
\end{definition}

A prestack $F$ is then a sheaf for the topology $\tau$ (here, $\tau$ is Zar or ét) if for every such covering family, $F(A)$ is the limit of the corresponding \v{C}ech diagram $F(B^{\otimes \bullet})$.

One can similarly define a hypersheaf by imposing a descent condition for hypercovers.

\begin{definition}
    A \emph{derived stack} is a prestack that is an étale hypersheaf.
\end{definition}

\begin{remark}
    To define modules on a prestack $\mathfrak{X}$ instead of just an animated ring $A$, one can construct the right Kan extension of $\Mod^* : \CAlg^{\ani, \op} \to \mathrm{Pr}^{\mathrm{L}, \otimes}$ to get a new functor
    \[
        \QCoh : \PreStack \to \mathrm{Pr}^{\mathrm{L}, \otimes}.
    \]
    By definition,
    \[
        \QCoh(\mathfrak{X}) \simeq \lim_{\Spec(B) \in \dSch^{\aff}_{/\mathfrak{X}}} \QCoh(\Spec(B))
    \]
    where $\QCoh(\Spec(B)) = \Mod_B$ is simply the $\infty$-category of animated $B$-modules.
\end{remark}

\begin{definition}
    A Zariski stack $\mathfrak{X}$ is a \emph{derived scheme} if there exists a cover by affine derived schemes : there exists a family $\mathcal{U} = (\mathcal{U}_{\alpha})$ with $\mathcal{U}_{\alpha} \to \mathfrak{X}$ open immersions from affine derived schemes $\mathcal{U}_{\alpha} \simeq \Spec(A_{\alpha})$, such that $\coprod_{\alpha} \mathcal{U}_{\alpha} \to \mathfrak{X}$ is an effective epimorphism.
\end{definition}

In this definition, \enquote{open immersion} refers to a morphism $\mathfrak{Y} \to \mathfrak{X}$ such that $\mathfrak{Y} \times_{\mathfrak{X}} \Spec(A) \to \Spec(A)$ is an open immersion for every animated ring $A$.
More generally, one can define representability in the following way.

\begin{definition}
    A morphism $f : \mathfrak{X} \to \mathfrak{Y}$ is \emph{(-1)-geometric} or \emph{representable} if for every map $\Spec(A) \to \mathfrak{Y}$, the pullback $\mathfrak{X} \times_{\mathfrak{Y}} \Spec(A)$ is an affine derived scheme $\Spec(B)$.
\end{definition}

The higher levels of geometricity are defined recursively.

\begin{definition}
\label{def geometricity}
    If $\mathfrak{X}$ is a derived stack, we say that it is \emph{(-1)-geometric} if it is affine.
    For every $n \geqslant 0$, we say that $\mathfrak{X}$ is \emph{$n$-geometric} if the following conditions hold :
    \begin{enumerate}
        \item The diagonal $\Delta : \mathfrak{X} \to \mathfrak{X} \times \mathfrak{X}$ is $(n-1)$-geometric ;
        \item There is an affine covering $\coprod \Spec(B_i) \twoheadrightarrow \mathfrak{X}$ that is $(n-1)$-geometric and étale.
    \end{enumerate}
    A morphism of derived stacks $f : \mathfrak{X} \to \mathfrak{Y}$ is \emph{$m$-geometric} for $m \geqslant 0$ if for every map $\Spec(A) \to \mathfrak{Y}$, the pullback $P$ in the following square
    % https://q.uiver.app/#q=WzAsNCxbMCwwLCJQIl0sWzEsMCwiXFxtYXRoZnJha3tYfSJdLFsxLDEsIlxcbWF0aGZyYWt7WX0iXSxbMCwxLCJcXFNwZWMoQSkiXSxbMSwyLCJmIl0sWzMsMl0sWzAsM10sWzAsMV0sWzAsMiwiIiwxLHsic3R5bGUiOnsibmFtZSI6ImNvcm5lciJ9fV1d
    \[\begin{tikzcd}
        P & {\mathfrak{X}} \\
        {\Spec(A)} & {\mathfrak{Y}}
        \arrow[from=1-1, to=1-2]
        \arrow[from=1-1, to=2-1]
        \arrow["\lrcorner"{anchor=center, pos=0.125}, draw=none, from=1-1, to=2-2]
        \arrow["f", from=1-2, to=2-2]
        \arrow[from=2-1, to=2-2]
    \end{tikzcd}\]
    is $m$-geometric.
\end{definition}

Along with the different levels of geometricity, there is also a hierarchy of levels of \emph{stack-ity}.

\begin{definition}
    A derived stack $\mathfrak{X}$ is an \emph{$n$-stack} if for every static ring $A \in \CAlg^{\heart}$, the space $\mathfrak{X}(A)$ is $n$-truncated.
\end{definition}
In other words, higher stacks have \enquote{higher coherence data}.

\begin{remark}
    A classical Artin stack is a 1-geometric 1-stack with these definitions.
    A \emph{higher Deligne-Mumford stack} would be a geometric stack with \enquote{smooth} replacing the word \emph{étale} in the second condition of Definition \ref{def geometricity}.
\end{remark}

\begin{proposition}
    The functor $\Perf : \CAlg^{\ani} \to \Spc$ sending an animated ring $A$ to the space of perfect complexes $\left(\Mod_A^{\mathrm{perf}}\right)^{\simeq}$ is a locally geometric stack : it is a filtered colimit of geometric stacks by monomorphic transition maps.
\end{proposition}
\begin{proof}[Ideas of the proof]
    This filtered colimit is none other than $\Perf = \bigcup_{a \leqslant b} \Perf_{[a, b]}$ where $\Perf_{[a, b]}$ denotes the perfect complexes with Tor-amplitude in $[a, b]$.

    One can show that $\Perf_{[a, b]}$ is $(b-a+1)$-geometric, by induction on $n = b-a+1$.
    The case $n = 1$ is already interesting, since $\Perf_{[a, a]} \simeq \mathrm{Vect} \simeq \coprod_m \mathrm{BGL}_m$.
    This is indeed 1-geometric, each $\mathrm{BGL}_m = [\ast / \mathrm{GL}_m]$ being a geometric realization.
\end{proof}
\end{document}