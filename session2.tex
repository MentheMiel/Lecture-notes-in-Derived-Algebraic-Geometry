\documentclass[11pt]{article}

\usepackage{quiver}
\usepackage{amsmath, amsthm, mathrsfs}
\usepackage[margin = 3cm]{geometry}
\usepackage{enumitem}
\usepackage{csquotes}

\newtheorem{theorem}{Theorem}
\newtheorem{proposition}{Proposition}
\newtheorem{lemma}{Lemma}
\theoremstyle{definition}
\newtheorem{definition}{Definition}
\newtheorem{remark}{Remark}
\newtheorem{example}{Example}
\newtheorem{exercise}{Exercise}

\newcommand{\A}{\mathbf{A}}
\newcommand{\C}{\mathcal{C}}
\newcommand{\CAlg}{\mathrm{CAlg}}
\newcommand{\Cart}{\operatorname{Cart}}
\newcommand{\Cat}{\mathrm{Cat}}
\newcommand{\Catinf}{\mathcal{C}\mathrm{at}_{\infty}}
\newcommand{\Cof}{\mathrm{Cof}}
\newcommand{\Cut}{\operatorname{Cut}}
\newcommand{\D}{\mathcal{D}}
\newcommand{\Fib}{\mathrm{Fib}}
\newcommand{\Fun}{\operatorname{Fun}}
\newcommand{\Grpd}{\mathrm{Grpd}}
\newcommand{\h}{\mathrm{h}}
\newcommand{\Hom}{\operatorname{Hom}}
\newcommand{\iHom}{\operatorname{\underline{Hom}}}
\newcommand{\J}{\mathcal{J}}
\newcommand{\Kan}{\mathrm{Kan}}
\renewcommand{\L}{\mathbb{L}}
\newcommand{\M}{\mathscr{M}}
\newcommand{\Map}{\operatorname{Map}}
\newcommand{\N}{\mathrm{N}}
\newcommand{\Ob}{\operatorname{Ob}}
\newcommand{\op}{\mathrm{op}}
\newcommand{\p}{\mathfrak{p}}
\newcommand{\QCoh}{\operatorname{QCoh}}
\newcommand{\Set}{\mathrm{Set}}
\newcommand{\Spc}{\mathcal{S}\mathrm{pc}}
\newcommand{\Spec}{\operatorname{Spec}}
\newcommand{\sSet}{\mathrm{sSet}}
\newcommand{\Tor}{\operatorname{Tor}}


\title{Lecture notes in Derived Algebraic Geometry\\Session 2}
\date{10 February 2025}
\author{Course by F. Binda, notes by E. Hecky}


\begin{document}
\maketitle

\section{More on limits in an $\infty$-category}

There is another way to define the notion of a (co)limit in an $\infty$-category, which looks a bit more like the 1-categorical notion.
Let $F : I \to \C$ be an element of $\Map(I, \C)$, with $I$ a simplicial set and $\C$ an $\infty$-category.
\begin{definition}
    A \emph{cone} over $F$ is a pair $(y, \eta)$ where $y \in \C$ and $\eta : \C_y \to F$ is a natural transformation (i.e. a map $\eta : I \times \Delta^1 \to C$ which restricts to $C_y : I \to \Delta^0 \xrightarrow{y} \C$ on $\{0\}$ and to $F$ on $\{1\}$).
\end{definition}

Let $(y, \eta)$ be a cone over $F$ and $x \in \C$.
Letting $c : \C \to \C^I$ be the functor taking an object $z \in \C$ to the constant diagram $c_z : I \to \C$ on $z$, we can define a map :
\[
    \Map_{\C}(x, y) \xrightarrow{c} \Map_{\Fun(I, \C)}(c_x, c_y) \xrightarrow{\eta_*} \Map_{\Fun(I, \C)}(c_x, F)
\]
up to a contractible choice for the composition with $\eta$.

\begin{proposition}
    A cone $(y, \eta)$ over $F$ is a limit cone for $F$ if for all $x \in \C$, the map above is a homotopy equivalence.
\end{proposition}

\begin{example}
    If $I$ is discrete, then $\Fun(I, \C) = \prod_I \C$.
    A diagram $F : I \to \C$ is then a collection of objects $\{y_i\}_{i \in I}$, and a cone over $F$ is a collection of maps of the form $\{y \xrightarrow{\pi_i} y_i\}_{i \in I}$.
    It is a limit cone if and only if $\Map(x, y) \simeq \prod_{i \in I} \Map(x, y_i)$.
\end{example}

\begin{lemma}
    Any two limits $(y, \eta)$ and $(y', \eta')$ for $F$ are equivalent.
\end{lemma}
\begin{proof}
    Since $(y', \eta')$ is a limit, we have $\Map(c_y, F) \simeq \Map(y, y')$.
    The image of $\eta$ gives a map $f : y \to y'$ such that $\eta' \circ f \simeq \eta$.
    Similarly, there is a map $g : y' \to y$ such that $\eta \circ g \simeq \eta'$.
    We then get
    \[
        \eta \circ g \circ f \simeq \eta' \circ f \simeq \eta
    \]
    so $g \circ f \simeq \mathrm{id}$. Similarly, $f \circ g \simeq \mathrm{id}$.
\end{proof}

\begin{remark}
    One can show the (much) stronger statement that the $\infty$-category of limit cones over a given functor is either empty or trivial (equivalent to $\{*\}$).
    This means that when a diagram has a limit, then the \enquote{space} of limits is contractible, whereas the lemma only showed that it is connected.
\end{remark}

\begin{exercise}
    Let $I = \{0 \to 1 \leftarrow 0'\}$ be the \enquote{pullback} diagram.
    A functor $F : I \to \C$ is equivalently the data of two maps $b \xrightarrow{h} d \xleftarrow{k} c$ in $\C$.
    Show that up to equivalence, the datum of a cone $\eta : c_a \to F$ is equivalent to the data of two morphisms $b \xleftarrow{i} a \xrightarrow{j} c$ and of a choice of equivalence $h \circ i \simeq k \circ j$.
\end{exercise}

\begin{proposition}
    A cone $(y, \eta)$ over $F : I \to \Spc$ is a limit cone if and only if for all $x \in \Spc$, the map
    \[
        [x, y]_{\Spc} = \pi_0 \Map(x, y) \to [c_x, F]_{\Spc^I} = \pi_0 \Map(c_x, F)
    \]
    is an equivalence.
\end{proposition}
In other words, the fact that a cone \emph{of spaces} is a limit can be checked in the homotopy category.
(This does not mean that the limit can be computed in the homotopy category.)

\begin{exercise}
    Check that the pullback of a diagram $b \to c \leftarrow d$ in $\Spc$ can be computed as the ordinary limit (iterated pullback) of the following diagram in $\sSet$ :
    % https://q.uiver.app/#q=WzAsNSxbMCwwLCJiIl0sWzEsMSwiYyJdLFsyLDAsImNee1xcRGVsdGFeMX0iXSxbMywxLCJjIl0sWzQsMCwiZCJdLFswLDFdLFsyLDEsIlxcbWF0aHJte2V2fV8wIiwyXSxbMiwzLCJcXG1hdGhybXtldn1fMSJdLFs0LDNdXQ==
    \[\begin{tikzcd}
        b && {c^{\Delta^1}} && d \\
        & c && c
        \arrow[from=1-1, to=2-2]
        \arrow["{\mathrm{ev}_0}"', from=1-3, to=2-2]
        \arrow["{\mathrm{ev}_1}", from=1-3, to=2-4]
        \arrow[from=1-5, to=2-4]
    \end{tikzcd}\]
\end{exercise}

\begin{proposition}
    Let $\C$ be a Kan-enriched category, $I$ a simplicial set and $F : I \to \N_{\Delta}(\C)$ a map to the simplicial nerve.
    Consider the functor $\iHom_{\C}(x, -) : \C \to \sSet$ and apply the simplicial nerve functor to get a functor
    \[
        \N_{\Delta}(\iHom(x, -)) : \N_{\Delta}(\C) \to \Spc.
    \]
    Then a cone $(y, \eta)$ is a limit cone for $F$ if and only if the cone $(\iHom(x, y), \N_{\Delta}(\iHom(x, \eta)))$ is a limit in $\Spc$.
\end{proposition}
In other words, \enquote{the $\Map$ functor preserves limits in the second argument} in the following way :
\[
    \Map_{\C}(x, \lim_I F) \simeq \lim_{i \in I} \Map(x, F(i)).
\]
The limit on the left is performed in $\C$, whereas the one on the right is performed in the $\infty$-category of spaces $\Spc$.

\begin{remark}
    A cone for $F$ is equivalent to the datum of a map
    \[
        (I \times \Delta^1)/(I \times \Delta^0) \to \C
    \]
    where the contracted $I \times \Delta^0$ is the copy of $I$ that is located at $\{0\}$.
    This quotient is another model for $I^{\triangleleft}$.
\end{remark}

Of course, one gets the notion of a colimit by dualizing this whole section.

\begin{theorem}
    Let $\C$ be an $\infty$-category.
    Then the following are equivalent :
    \begin{enumerate}
        \item $\C$ has an initial object and pushouts ;
        \item $\C$ has finite coproducts and coequalizers ;
        \item $\C$ has finite limits.
    \end{enumerate}
    Moreover, if $\C$ has arbitrary coproducts, then the following are equivalent :
    \begin{enumerate}
        \item $\C$ has all colimits ;
        \item $\C$ has pushouts ;
        \item $\C$ has coequalizers ;
        \item $\C$ has geometric realizations (colimits of diagrams of shape $\Delta^{\op}$).
    \end{enumerate}
\end{theorem}

\end{document}