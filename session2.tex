\documentclass[11pt]{article}

\usepackage{quiver}
\usepackage{amsmath, amsthm, mathrsfs}
\usepackage[margin = 3cm]{geometry}
\usepackage{enumitem}
\usepackage{csquotes}

\newtheorem{theorem}{Theorem}
\newtheorem{proposition}{Proposition}
\newtheorem{lemma}{Lemma}
\newtheorem{corollary}{Corollary}
\theoremstyle{definition}
\newtheorem{definition}{Definition}
\newtheorem{remark}{Remark}
\newtheorem{example}{Example}
\newtheorem{exercise}{Exercise}

\newcommand{\A}{\mathbf{A}}
\newcommand{\Ani}{\operatorname{Ani}}
\newcommand{\ani}{\mathrm{ani}}
\newcommand{\C}{\mathcal{C}}
\newcommand{\CAlg}{\mathrm{CAlg}}
\newcommand{\Cart}{\operatorname{Cart}}
\newcommand{\Cat}{\mathrm{Cat}}
\newcommand{\Catinf}{\mathcal{C}\mathrm{at}_{\infty}}
\newcommand{\cc}{\mathrm{cc}}
\newcommand{\Cof}{\mathrm{Cof}}
\newcommand{\colim}{\operatorname*{colim}}
\newcommand{\Cut}{\operatorname{Cut}}
\newcommand{\D}{\mathcal{D}}
\newcommand{\E}{\mathcal{E}}
\newcommand{\Eq}{\operatorname{Eq}}
\newcommand{\Fib}{\mathrm{Fib}}
\newcommand{\Fun}{\operatorname{Fun}}
\newcommand{\Grpd}{\mathrm{Grpd}}
\newcommand{\h}{\mathrm{h}}
\newcommand{\heart}{\heartsuit}
\newcommand{\Hom}{\operatorname{Hom}}
\newcommand{\iHom}{\operatorname{\underline{Hom}}}
\newcommand{\Ind}{\operatorname{Ind}}
\newcommand{\J}{\mathcal{J}}
\newcommand{\Kan}{\mathrm{Kan}}
\renewcommand{\L}{\mathbb{L}}
\newcommand{\M}{\mathscr{M}}
\newcommand{\Map}{\operatorname{Map}}
\newcommand{\N}{\mathrm{N}}
\newcommand{\Ob}{\operatorname{Ob}}
\newcommand{\op}{\mathrm{op}}
\newcommand{\p}{\mathfrak{p}}
\renewcommand{\P}{\mathcal{P}}
\newcommand{\Poly}{\mathrm{Poly}}
\newcommand{\proj}{\mathrm{proj}}
\newcommand{\QCoh}{\operatorname{QCoh}}
\newcommand{\Set}{\mathrm{Set}}
\newcommand{\Sp}{\mathcal{S}\mathrm{p}}
\newcommand{\Spc}{\mathcal{S}\mathrm{pc}}
\newcommand{\Spec}{\operatorname{Spec}}
\newcommand{\sSet}{\mathrm{sSet}}
\newcommand{\Tor}{\operatorname{Tor}}
\newcommand{\X}{\mathcal{X}}
\newcommand{\Y}{\mathcal{Y}}
\newcommand{\Z}{\mathbf{Z}}


\title{Lecture notes in Derived Algebraic Geometry\\Session 2}
\date{10 February 2025}
\author{Course by F. Binda, notes by E. Hecky}


\begin{document}
\maketitle

\section{More on limits in an $\infty$-category}

There is another way to define the notion of a (co)limit in an $\infty$-category, which looks a bit more like the 1-categorical notion.
Let $F : I \to \C$ be an element of $\Map(I, \C)$, with $I$ a simplicial set and $\C$ an $\infty$-category.
\begin{definition}
    A \emph{cone} over $F$ is a pair $(y, \eta)$ where $y \in \C$ and $\eta : \C_y \to F$ is a natural transformation (i.e. a map $\eta : I \times \Delta^1 \to C$ which restricts to $C_y : I \to \Delta^0 \xrightarrow{y} \C$ on $\{0\}$ and to $F$ on $\{1\}$).
\end{definition}

Let $(y, \eta)$ be a cone over $F$ and $x \in \C$.
Letting $c : \C \to \C^I$ be the functor taking an object $z \in \C$ to the constant diagram $c_z : I \to \C$ on $z$, we can define a map :
\[
    \Map_{\C}(x, y) \xrightarrow{c} \Map_{\Fun(I, \C)}(c_x, c_y) \xrightarrow{\eta_*} \Map_{\Fun(I, \C)}(c_x, F)
\]
up to a contractible choice for the composition with $\eta$.

\begin{proposition}
    A cone $(y, \eta)$ over $F$ is a limit cone for $F$ if for all $x \in \C$, the map above is a homotopy equivalence.
\end{proposition}

\begin{example}
    If $I$ is discrete, then $\Fun(I, \C) = \prod_I \C$.
    A diagram $F : I \to \C$ is then a collection of objects $\{y_i\}_{i \in I}$, and a cone over $F$ is a collection of maps of the form $\{y \xrightarrow{\pi_i} y_i\}_{i \in I}$.
    It is a limit cone if and only if $\Map(x, y) \simeq \prod_{i \in I} \Map(x, y_i)$.
\end{example}

\begin{lemma}
    Any two limits $(y, \eta)$ and $(y', \eta')$ for $F$ are equivalent.
\end{lemma}
\begin{proof}
    Since $(y', \eta')$ is a limit, we have $\Map(c_y, F) \simeq \Map(y, y')$.
    The image of $\eta$ gives a map $f : y \to y'$ such that $\eta' \circ f \simeq \eta$.
    Similarly, there is a map $g : y' \to y$ such that $\eta \circ g \simeq \eta'$.
    We then get
    \[
        \eta \circ g \circ f \simeq \eta' \circ f \simeq \eta
    \]
    so $g \circ f \simeq \mathrm{id}$. Similarly, $f \circ g \simeq \mathrm{id}$.
\end{proof}

\begin{remark}
    One can show the (much) stronger statement that the $\infty$-category of limit cones over a given functor is either empty or trivial (equivalent to $\{*\}$).
    This means that when a diagram has a limit, then the \enquote{space} of limits is contractible, whereas the lemma only showed that it is connected.
\end{remark}

\begin{exercise}
    Let $I = \{0 \to 1 \leftarrow 0'\}$ be the \enquote{pullback} diagram.
    A functor $F : I \to \C$ is equivalently the data of two maps $b \xrightarrow{h} d \xleftarrow{k} c$ in $\C$.
    Show that up to equivalence, the datum of a cone $\eta : c_a \to F$ is equivalent to the data of two morphisms $b \xleftarrow{i} a \xrightarrow{j} c$ and of a choice of equivalence $h \circ i \simeq k \circ j$.
\end{exercise}

\begin{proposition}
    A cone $(y, \eta)$ over $F : I \to \Spc$ is a limit cone if and only if for all $x \in \Spc$, the map
    \[
        [x, y]_{\Spc} = \pi_0 \Map(x, y) \to [c_x, F]_{\Spc^I} = \pi_0 \Map(c_x, F)
    \]
    is an equivalence.
\end{proposition}
In other words, the fact that a cone \emph{of spaces} is a limit can be checked in the homotopy category.
(This does not mean that the limit can be computed in the homotopy category.)

\begin{exercise}
    Check that the pullback of a diagram $b \to c \leftarrow d$ in $\Spc$ can be computed as the ordinary limit (iterated pullback) of the following diagram in $\sSet$ :
    % https://q.uiver.app/#q=WzAsNSxbMCwwLCJiIl0sWzEsMSwiYyJdLFsyLDAsImNee1xcRGVsdGFeMX0iXSxbMywxLCJjIl0sWzQsMCwiZCJdLFswLDFdLFsyLDEsIlxcbWF0aHJte2V2fV8wIiwyXSxbMiwzLCJcXG1hdGhybXtldn1fMSJdLFs0LDNdXQ==
    \[\begin{tikzcd}
        b && {c^{\Delta^1}} && d \\
        & c && c
        \arrow[from=1-1, to=2-2]
        \arrow["{\mathrm{ev}_0}"', from=1-3, to=2-2]
        \arrow["{\mathrm{ev}_1}", from=1-3, to=2-4]
        \arrow[from=1-5, to=2-4]
    \end{tikzcd}\]
\end{exercise}

\begin{proposition}
    Let $\C$ be a Kan-enriched category, $I$ a simplicial set and $F : I \to \N_{\Delta}(\C)$ a map to the simplicial nerve.
    Consider the functor $\iHom_{\C}(x, -) : \C \to \sSet$ and apply the simplicial nerve functor to get a functor
    \[
        \N_{\Delta}(\iHom(x, -)) : \N_{\Delta}(\C) \to \Spc.
    \]
    Then a cone $(y, \eta)$ is a limit cone for $F$ if and only if the cone $(\iHom(x, y), \N_{\Delta}(\iHom(x, \eta)))$ is a limit in $\Spc$.
\end{proposition}
In other words, \enquote{the $\Map$ functor preserves limits in the second argument} in the following way :
\[
    \Map_{\C}(x, \lim_I F) \simeq \lim_{i \in I} \Map(x, F(i)).
\]
The limit on the left is performed in $\C$, whereas the one on the right is performed in the $\infty$-category of spaces $\Spc$.

\begin{remark}
    A cone for $F$ is equivalent to the datum of a map
    \[
        (I \times \Delta^1)/(I \times \Delta^0) \to \C
    \]
    where the contracted $I \times \Delta^0$ is the copy of $I$ that is located at $\{0\}$.
    This quotient is another model for $I^{\triangleleft}$.
\end{remark}

Of course, one gets the notion of a colimit by dualizing this whole section.

\begin{theorem}
    Let $\C$ be an $\infty$-category.
    Then the following are equivalent :
    \begin{enumerate}
        \item $\C$ has an initial object and pushouts ;
        \item $\C$ has finite coproducts and coequalizers ;
        \item $\C$ has finite limits.
    \end{enumerate}
    Moreover, if $\C$ has arbitrary coproducts, then the following are equivalent :
    \begin{enumerate}
        \item $\C$ has all colimits ;
        \item $\C$ has pushouts ;
        \item $\C$ has coequalizers ;
        \item $\C$ has geometric realizations (colimits of diagrams of shape $\Delta^{\op}$).
    \end{enumerate}
\end{theorem}

\section{Cartesian fibrations}

Let $p : \X \to \Y$ be a functor of $\infty$-categories modelled by an inner fibration of simplicial sets. It means that it has the right lifting property with respect to every inner horn inclusion $\Lambda_k^n \hookrightarrow \Delta^n$ : every square of solid arrows of the following form has a dashed lift
% https://q.uiver.app/#q=WzAsNCxbMCwwLCJcXExhbWJkYV9rXm4iXSxbMCwxLCJcXERlbHRhXm4iXSxbMSwwLCJcXFgiXSxbMSwxLCJcXFkiXSxbMiwzLCJwIl0sWzAsMl0sWzAsMSwiIiwyLHsic3R5bGUiOnsidGFpbCI6eyJuYW1lIjoiaG9vayIsInNpZGUiOiJ0b3AifX19XSxbMSwzXSxbMSwyLCIiLDEseyJzdHlsZSI6eyJib2R5Ijp7Im5hbWUiOiJkYXNoZWQifX19XV0=
\[\begin{tikzcd}
	{\Lambda_k^n} & \X \\
	{\Delta^n} & \Y
	\arrow[from=1-1, to=1-2]
	\arrow[hook, from=1-1, to=2-1]
	\arrow["p", from=1-2, to=2-2]
	\arrow[dashed, from=2-1, to=1-2]
	\arrow[from=2-1, to=2-2]
\end{tikzcd}\]
for all $0 < k < n$.
Let $f : \Delta^1 \to \X$ be a an arrow in $\X$.

\begin{definition}
    The map $f$ is \emph{$p$-cartesian} if every diagram of solid arrows of the following form has a dashed lift :
    % https://q.uiver.app/#q=WzAsNSxbMSwwLCJcXExhbWJkYV9uXm4iXSxbMSwxLCJcXERlbHRhXm4iXSxbMiwwLCJcXFgiXSxbMiwxLCJcXFkiXSxbMCwwLCJcXERlbHRhXjEgPSBcXERlbHRhXntuLTEsIG59Il0sWzIsMywicCJdLFsxLDNdLFswLDEsIiIsMix7InN0eWxlIjp7InRhaWwiOnsibmFtZSI6Imhvb2siLCJzaWRlIjoidG9wIn19fV0sWzAsMl0sWzEsMiwiIiwxLHsic3R5bGUiOnsiYm9keSI6eyJuYW1lIjoiZGFzaGVkIn19fV0sWzQsMCwiIiwxLHsic3R5bGUiOnsidGFpbCI6eyJuYW1lIjoiaG9vayIsInNpZGUiOiJ0b3AifX19XSxbNCwyLCJmIiwwLHsiY3VydmUiOi0yfV1d
    \[\begin{tikzcd}
        {\Delta^1 = \Delta^{\{n-1, n\}}} & {\Lambda_n^n} & \X \\
        & {\Delta^n} & \Y
        \arrow[hook, from=1-1, to=1-2]
        \arrow["f", curve={height=-15pt}, from=1-1, to=1-3]
        \arrow[from=1-2, to=1-3]
        \arrow[hook, from=1-2, to=2-2]
        \arrow["p", from=1-3, to=2-3]
        \arrow[dashed, from=2-2, to=1-3]
        \arrow[from=2-2, to=2-3]
    \end{tikzcd}\]
    Similarly, $f$ is \emph{$p$-cocartesian} if the same condition is true with $\Delta^{\{0, 1\}} \hookrightarrow \Lambda_0^n$ instead.
\end{definition}

\begin{lemma}
    $f$ is $p$-cartesian if and only if the map
    \[
        \X_{/f} \to \X_{/y} \times_{\Y_{/p(y)}} \Y_{/p(f)}
    \]
    is a trivial fibration of simplicial sets.
    (A trivial fibration is a map that has the right lifting property with respect to every boundary inclusion $\partial \Delta^n \hookrightarrow \Delta^n$.)
\end{lemma}

\begin{definition}
    A functor $p : \X \to \Y$ is a \emph{Cartesian fibration} if it is an inner fibration such that every lifting problem
    % https://q.uiver.app/#q=WzAsNCxbMCwwLCJcXERlbHRhXjAiXSxbMCwxLCJcXERlbHRhXjEiXSxbMSwwLCJcXFgiXSxbMSwxLCJcXFkiXSxbMiwzLCJwIl0sWzAsMl0sWzAsMSwiMSIsMl0sWzEsM10sWzEsMiwiIiwxLHsic3R5bGUiOnsiYm9keSI6eyJuYW1lIjoiZGFzaGVkIn19fV1d
    \[\begin{tikzcd}
        {\Delta^0} & \X \\
        {\Delta^1} & \Y
        \arrow[from=1-1, to=1-2]
        \arrow["1"', from=1-1, to=2-1]
        \arrow["p", from=1-2, to=2-2]
        \arrow[dashed, from=2-1, to=1-2]
        \arrow[from=2-1, to=2-2]
    \end{tikzcd}\]
    has a solution $\Delta^1 \to \X$ which is $p$-cartesian.
    Dually, $p$ is a \emph{coCartesian fibration} if it is an inner fibration such that every lifting problem
    % https://q.uiver.app/#q=WzAsNCxbMCwwLCJcXERlbHRhXjAiXSxbMCwxLCJcXERlbHRhXjEiXSxbMSwwLCJcXFgiXSxbMSwxLCJcXFkiXSxbMiwzLCJwIl0sWzAsMl0sWzAsMSwiMCIsMl0sWzEsM10sWzEsMiwiIiwxLHsic3R5bGUiOnsiYm9keSI6eyJuYW1lIjoiZGFzaGVkIn19fV1d
    \[\begin{tikzcd}
        {\Delta^0} & \X \\
        {\Delta^1} & \Y
        \arrow[from=1-1, to=1-2]
        \arrow["0"', from=1-1, to=2-1]
        \arrow["p", from=1-2, to=2-2]
        \arrow[dashed, from=2-1, to=1-2]
        \arrow[from=2-1, to=2-2]
    \end{tikzcd}\]
    has a solution $\Delta^1 \to \X$ which is $p$-cocartesian.
\end{definition}

A Cartesian (resp. coCartesian) fibration is a tractable way of constructing a contravariant (resp. covariant) functor with values in $\infty$-categories, via the following straightening construction.

Let $p : \E \to \C$ be a coCartesian fibration.
\begin{enumerate}
    \item For all $x \in \C$, let $\E_x$ be the fiber $\E \times_{\C} \Delta^0$ over $x$.
    It is automatically a weak Kan complex since $p$ is an inner fibration.
    \item Let $f : x \to y$ be a morphism in $\C$, we want to construct an associated functor $f_! : \E_x \to \E_y$.
    Let $z \in \E_x$.
    Since $p(z) = x$, we have a morphism $f : p(z) \to y$ in $\C$, so a square
    % https://q.uiver.app/#q=WzAsNCxbMCwwLCJcXERlbHRhXjAiXSxbMCwxLCJcXERlbHRhXjEiXSxbMSwwLCJcXEUiXSxbMSwxLCJcXEMiXSxbMCwxLCIwIiwyXSxbMiwzLCJwIl0sWzAsMiwieiJdLFsxLDMsImYiLDJdXQ==
    \[\begin{tikzcd}
        {\Delta^0} & \E \\
        {\Delta^1} & \C
        \arrow["z", from=1-1, to=1-2]
        \arrow["0"', from=1-1, to=2-1]
        \arrow["p", from=1-2, to=2-2]
        \arrow["f"', from=2-1, to=2-2]
    \end{tikzcd}\]
    There \emph{exists} a $p$-cocartesian lift $\Delta^1 \to \E$, seen as a map $z \to w$ with $w \in \E_y$.
    We want to define $f_!(z) = w$ : how ambiguous is this construction ?
    \item Given $z' \in \E_x$ and $\alpha : z \to z'$, we can choose another $p$-cocartesian lift $z' \to w' = f_!(z')$.
    By $p$-cocartesianity, the following diagram can be extended with a dashed arrow, giving the desired equivalence :
    % https://q.uiver.app/#q=WzAsNCxbMCwwLCJ6Il0sWzEsMSwidyciXSxbMSwwLCJ6JyJdLFswLDEsInciXSxbMCwyLCJcXGFscGhhIl0sWzIsMV0sWzAsM10sWzAsMV0sWzMsMSwiIiwyLHsic3R5bGUiOnsiYm9keSI6eyJuYW1lIjoiZGFzaGVkIn19fV1d
    \[\begin{tikzcd}
        z & {z'} \\
        w & {w'}
        \arrow["\alpha", from=1-1, to=1-2]
        \arrow[from=1-1, to=2-1]
        \arrow[from=1-1, to=2-2]
        \arrow[from=1-2, to=2-2]
        \arrow[dashed, from=2-1, to=2-2]
    \end{tikzcd}\]
\end{enumerate}

\begin{proposition}
    Let $p : \E \to \C$ be a coCartesian fibration and $f : \Delta^1 \to \C$ be a map $x \to y$.
    Then there exists a functor $\E_x \times \Delta^1 \to \E$ such that
    \begin{enumerate}
        \item For every $z \in \E_x$, restricting to $z$ gives an arrow $\{z\} \times \Delta^1 \to \E$ of the form $\beta : z \to z'$ which is a $p$-cocartesian lift of $f$ ;
        \item The restriction to $\E_x \times \{1\}$ yields the desired functor $f_! : \E_x \to \E_y$.
    \end{enumerate}
\end{proposition}
\begin{proof}[Ideas of the proof]
    Let $\Fun_f(\Delta^1, \E) = \{g : \Delta^1 \to \E \mid p \circ g \simeq f\}$ and $\Fun_f^{\cc}(\Delta^1, \E)$ be the subcategory on the $p$-cocartesian lifts.
    There is a functor $\Fun_f^{\cc}(\Delta^1, \E) \to \E_x$ taking the source, and it is a trivial fibration.
    Choose a section $\E_x \to \Fun_f^{\cc}(\Delta^1, \E)$ : by adjunction, it gives the functor announced in the statement.
    Composing this section with the target functor yields the desired functor :
    \[
        f_! : \E_x \to \Fun_f^{\cc}(\Delta^1, \E) \xrightarrow{\text{target}} \E_y.
    \]
\end{proof}

More generally, for every simplex $\sigma : \Delta^n \to \C$ one can define $\Fun_{\sigma}^{\cc}(\Delta^n, \E)$, and then a functor
\[
    \vartheta(p) : \Delta^{\op}_{/\C} \to \sSet
\]
associating to $\sigma : \Delta^n \to \C$ the $\infty$-category $\Fun_{\sigma}^{\cc}(\Delta^n, \E)$.
Denoting by $W_{\C}$ the set of morphisms $f : [n] \to [m]$ in $\Delta_{\/C}$ such that $f(0) = 0$, one can show that $\vartheta(p)$ sends its elements to Joyal equivalences.

\begin{corollary}
    For every coCartesian fibration $p : \E \to \C$, there is a functor
    \[
        \Theta(p) : \N(\Delta^{\op}_{/\C})[W_{\C}^{-1}] \to \Catinf
    \]
    and the map from the source category to $\C$ taking the initial vertex is a Joyal equivalence, whence a functor $\C \to \Catinf$.
\end{corollary}
This is the straightening of $p$.
We finally obtain the un/straightening theorem that was needed to construct functors with values in $\infty$-categories.
\begin{theorem}
    For every $\infty$-category $\C$, there are equivalences of $\infty$-categories
    \[
        \Fun(\C, \Catinf) \simeq \operatorname{CoCart}(\C)
    \]
    and
    \[
        \Fun(\C^{\op}, \Catinf) \simeq \operatorname{Cart}(\C).
    \]
\end{theorem}

Cartesian fibrations also give a way to compute limits of diagrams of $\infty$-categories.
\begin{theorem}
    Let $F : \J \to \Catinf$ be an $\infty$-functor, i.e. a diagram of $\infty$-categories.
    Let $\int F \to \J^{\op}$ be the associated Cartesian fibration.
    Then the limit of $F$ exists and can be computed as the $\infty$-category of cartesian sections (that is, functors that send edges to cartesian maps) :
    \[
        \lim_{\J} F \simeq \Fun_{/\J^{\op}}^{\mathrm{Cart}}(\J^{\op}, \int F).
    \]
\end{theorem}

\section{Cocompletions}

\subsection{The 1-categorical story}

Let $C$ be an ordinary 1-category.
There is a Yoneda embedding $C \to \P(C)^{\heart}$ into the 1-category of presheaves of sets on $C$.
It has the universal property that $\Fun'(\P(C)^{\heart}, D) \simeq \Fun(C, D)$ for every cocomplete category $D$, the notation $\Fun'$ denoting the subcategory of functors that preserve colimits.

\begin{exercise}
    Show that the Yoneda functor does not preserve coproducts.
\end{exercise}

This is a problem, but it can be corrected by considering sifted colimits.
\begin{definition}
    A 1-category $C$ is \emph{1-sifted} if it is not empty and if $C$-indexed colimits in $\Set$ commute with finite products.
\end{definition}

\begin{exercise}
    Show that $C$ is 1-sifted if and only if it is not empty and the diagonal map $\Delta : C \to C \times C$ is cofinal.
\end{exercise}

\begin{example}
    The category $\Delta^{\op}_{\leqslant 1}$ modelling reflexive coequalizers is 1-sifted.
    Note that the category $\{[1] \rightrightarrows [0]\}$ modelling coequalizers is not 1-sifted.
\end{example}

\begin{definition}
    Let $C$ be a 1-category with finite coproducts.
    Define $\P_{\Sigma}(C)^{\heart}$ as the full subcategory of $\P(C)^{\heart}$ on functors preserving finite coproducts (since they are contravariant, it means that they map finite sums to products).
\end{definition}

\begin{proposition}
    \begin{enumerate}
        \item The inclusion $\P_{\Sigma}(C)^{\heart} \hookrightarrow \P(C)^{\heart}$ preserves 1-sifted colimits and has a left adjoint.
        In particular, $\P_{\Sigma}(C)^{\heart}$ is a reflexive localization of $\P(C)^{\heart}$, so it is presentable ;
        \item The Yoneda embedding $j : C \to \P(C)^{\heart}$ lands in $\P_{\Sigma}(C)^{\heart}$ ;
        \item $j : C \to \P_{\Sigma}(C)^{\heart}$ preserves finite coproducts ;
        \item A presheaf $X$ is in $\P_{\Sigma}(C)^{\heart}$ if and only if it can be written as a reflexive coequalizer of filtered colimits of representables.
    \end{enumerate}
    Denoting by $\Fun''$ the category of sifted-colimit preserving functors and by $\Fun^{\amalg}$ the category of coproduct-preserving functors, we also have the universal properties
    \[
        \Fun''(\P_{\Sigma}(C)^{\heart}, D) \simeq \Fun(C, D)
    \]
    and
    \[
        \Fun^{\amalg}(C, D) \simeq \Fun'(\P_{\Sigma}(C)^{\heart}, D)
    \]
    for every category $D$ with colimits.
\end{proposition}
How does the story extend to $\infty$-categories ?

\subsection{The $\infty$-categorical version}

Let $\C$ be a an $\infty$-category and $\P(\C) = \Fun(\C^{\op}, \Spc)$ be its presheaf $\infty$-category.
Then $\P(\C)$ has all colimits, and there is again an equivalence
\[
    \Fun(\C, \D) \simeq \Fun'(\P(\C), \D)
\]
for cocomplete $\infty$-categories $\D$.

\begin{definition}
    An $\infty$-category $K \in \Catinf$ is \emph{sifted} if $K$-indexed colimits in $\Spc$ commute with finite products.
\end{definition}

\begin{example}
    $\Delta^{\op}$ is sifted.
    Its diagonal is cofinal ; it is well-known that if $X_{\bullet, \bullet}$ is a bisimplicial set then its geometric realization can be computed as the colimit of the diagonal $X_{n, n}$.
    On the other hand, the 1-sifted category $\Delta^{\op}_{\leqslant 1}$ is not sifted.
\end{example}

\begin{definition}
    Let $\C$ be an $\infty$-category with finite coproducts.
    Then $\P_{\Sigma}(\C)$ is defined as the full subcategory of $\P(\C)$ on functors that preserve products.
\end{definition}

The same proposition as in the 1-categorical picture holds.
\begin{proposition}
    \begin{enumerate}
        \item The inclusion $\P_{\Sigma}(\C) \hookrightarrow \P(\C)$ preserves sifted colimits and has a left adjoint.
        In particular, $\P_{\Sigma}(\C)$ is presentable, and it has limits and colimits ;
        \item The Yoneda embedding $j : \C \to \P(\C)$ lands in $\P_{\Sigma}(\C)$ ;
        \item $j : \C \to \P_{\Sigma}(\C)$ preserves finite coproducts ;
        \item A presheaf $X$ is in $\P_{\Sigma}(\C)$ if and only if it is the geometric realization of filtered colimits of representables.
    \end{enumerate}
    Also, we have the universal properties
    \[
        \Fun''(\P_{\Sigma}(\C), \D) \simeq \Fun(\C, \D)
    \]
    and
    \[
        \Fun^{\amalg}(\C, \D) \simeq \Fun'(\P_{\Sigma}(\C), \D).
    \]
    In particular, any functor $\C \to \D$ can be extended to a sifted-colimit-preserving functor $\P_{\Sigma}(\C) \to \D$.
\end{proposition}

\section{Animated rings}

There are now several equivalent ways to define derived schemes.
The classical picture modelled them on simplicial commutative rings, but here we will prefer animated rings.

\subsection{Compact projective objects}

\begin{definition}
Let $\C$ be an $\infty$-category.
An object $x \in \C$ is \emph{compact projective} (resp. \emph{compact}) if the functor $\Map_{\C}(x, -) : \C \to \Spc$ commutes with sifted (resp. filtered) colimits.    
\end{definition}
Since a filtered colimit is a particular case of a sifted colimit, compact projective objects are compact.
Denote by $\C^{\omega}$ the $\infty$-category of compact objects in $\C$, and by $\C^{\omega\proj}$ the $\infty$-category of compact projective objects in $\C$.

\begin{definition}
    An $\infty$-category $\C$ is \emph{compactly generated} if the natural map $\Ind(\C^{\omega}) \to \C$ is an equivalence.
    It is \emph{generated by compact projective objects} if the natural map $\P_{\Sigma}(\C^{\omega\proj}) \to \C$ is an equivalence.
\end{definition}

\begin{remark}
    One can also define \emph{1-compact projective} objects in a 1-category $C$ as those $x \in C$ such that $\Hom_{C}(x, -)$ preserves 1-sifted colimits.
    $C$ is then said to be \emph{generated by 1-compact projective objects} if the natural map $\P_{\Sigma}(C^{\omega\mathrm{1-proj}}) \to C$ is an equivalence.
\end{remark}

\begin{definition}
    Let $C$ be an 1-category generated by its 1-compact projective objects.
    Then define its \emph{animation} $\Ani(C)$ as the $\infty$-category $\P_{\Sigma}(C^{\omega\proj}) \subset \Fun(\N(C)^{\omega\proj, \op}, \Spc)$.
\end{definition}

\subsection{Animated algebras}

\begin{example}
    Let $R$ be a commutative ring.
    Denote by $\Ani(R)$ or $\CAlg_R^{\ani}$ the animation of the category of $R$-algebras.
\end{example}
What are the 1-compact projective algebras ?

\begin{lemma}
    Let $R$ be a commutative ring.
    Then $R$ is 1-compact projective (aka strongly finitely presented) if and only if it is a retract of a finite polynomial algebra over $\Z$.
\end{lemma}
\begin{proof}
    The free $\dashv$ forgetful adjunction $\Set \rightleftarrows \CAlg_{\Z}$ gives a reflexive coequalizer diagram
    % https://q.uiver.app/#q=WzAsMyxbMiwwLCJcXFpbXFx1bmRlcmxpbmV7Un1dIl0sWzQsMCwiUiJdLFswLDAsIlxcWltcXFpbXFx1bmRlcmxpbmV7Un1dXSJdLFswLDJdLFsyLDAsIiIsMCx7Im9mZnNldCI6LTN9XSxbMiwwLCIiLDEseyJvZmZzZXQiOjN9XSxbMCwxLCJcXHZhcmVwc2lsb24iXV0=
    \[\begin{tikzcd}
        {\Z[\Z[\underline{R}]]} && {\Z[\underline{R}]} && R
        \arrow[shift left=3, from=1-1, to=1-3]
        \arrow[shift right=3, from=1-1, to=1-3]
        \arrow[from=1-3, to=1-1]
        \arrow["\varepsilon", from=1-3, to=1-5]
    \end{tikzcd}\]
    Letting $S_1 = \Z[\Z[\underline{R}]]$ and $S_0 = \Z[\underline{R}]$, one shows that the coequalizer $\colim(S_1 \rightrightarrows S_0)$ is isomorphic to $R$.
    If $R$ is 1-compact projective, then we have
    \begin{align*}
        &\colim(\Hom(R, S_1) \rightrightarrows \Hom(R, S_0))\\
        &\simeq \Hom(R, \colim(S_1 \rightrightarrows S_0))\\
        &\simeq \Hom(R, R).
    \end{align*}
    The image of the identity gives a section of $\varepsilon$, so $R$ is a retract of the polynomial algebra $\Z[\underline{R}]$.
    Also, $\Z[\underline{R}] = \colim_{F \subset \underline{R}\text{ finite}}\Z[F]$ and since $R$ is compact it factors through a $\Z[F]$. In particular, it is a retract of a \emph{finite} polynomial algebra.
\end{proof}

\begin{remark}
    Animated commutative algebras can therefore be described in a more concise way as :
    \[
        \CAlg_R^{\ani} \simeq \P_{\Sigma}(\Poly_R)
    \]
    where $\Poly_R$ is the category of finite polynomial algebras over $R$.
\end{remark}

\begin{remark}
    As explained in the previous subsection, any animated $R$-algebra is the geometric realization of a simplicial object $X_{\bullet}$, where each $X_n$ is a filtered colimit of finitely generated polynomial algebras.
\end{remark}

\begin{definition}
    An animated algebra $S \in \Ani(R)$ is \emph{$n$-truncated} if for every $X \in \Poly_R$, the space $\Map_{\Ani(R)}(X, S)$ is $n$-truncated. (Recall that being $n$-truncated means that the homotopy groups of order higher than $n$ vanish.)
    This defines a full subcategory $\tau_{\leqslant n} \Ani(R) \hookrightarrow \Ani(R)$.
\end{definition}
This full inclusion has a left adjoint $\tau_{\leqslant n}$, given by composition with the usual truncation functor on spaces :
\[
    \tau_{\leqslant n} S : \Poly_R^{\op} \xrightarrow{S} \Spc \xrightarrow{\tau_{\leqslant n}} \Spc_{\leqslant n} \hookrightarrow \Spc.
\]
This composition still preserves finite products, so it is still an animated algebra ; and the functor $\tau_{\leqslant n} : \Ani(R) \to \Ani(R)$ that we obtain indeed lands in $\tau_{\leqslant n}\Ani(R)$.

\begin{definition}
    For an animated $R$-algebra $S$, define its \emph{homotopy ring} as
    \[
        \pi_*(S) = \pi_*(\Map_{\Ani(R)}(R[t], S)).
    \]
    It is a graded commutative ring.
\end{definition}

\begin{definition}
    If $S \in \Ani(R)$, then the space $S(R[t]) = \Map_{\Ani(R)}(R[t], S) \in \Spc$ is called the \emph{underlying space} of $S$.
\end{definition}

\begin{remark}
    There is a fully faithful functor
    \[
        \CAlg_R^{\heart} \hookrightarrow \Ani(R)
    \]
    which is right adjoint to the functor $\pi_0 : \CAlg_R^{\ani} \to \CAlg_R^{\heart}$.
\end{remark}

\section{Spectra}

Recall that the suspension and loop space functors $\Sigma, \Omega : \Spc_* \to \Spc_*$ defined as the following pushout and pullback
% https://q.uiver.app/#q=WzAsOCxbMCwwLCJYIl0sWzMsMSwiWCJdLFsxLDAsIioiXSxbMCwxLCIqIl0sWzMsMCwiKiJdLFsyLDEsIioiXSxbMiwwLCJcXE9tZWdhIFgiXSxbMSwxLCJcXFNpZ21hIFgiXSxbMCwyXSxbMCwzXSxbNiw0XSxbNiw1XSxbMyw3XSxbMiw3XSxbNSwxXSxbNCwxXSxbNywwLCIiLDEseyJzdHlsZSI6eyJuYW1lIjoiY29ybmVyIn19XSxbNiwxLCIiLDEseyJzdHlsZSI6eyJuYW1lIjoiY29ybmVyIn19XV0=
\[\begin{tikzcd}
	X & {*} & {\Omega X} & {*} \\
	{*} & {\Sigma X} & {*} & X
	\arrow[from=1-1, to=1-2]
	\arrow[from=1-1, to=2-1]
	\arrow[from=1-2, to=2-2]
	\arrow[from=1-3, to=1-4]
	\arrow[from=1-3, to=2-3]
	\arrow["\lrcorner"{anchor=center, pos=0.125}, draw=none, from=1-3, to=2-4]
	\arrow[from=1-4, to=2-4]
	\arrow[from=2-1, to=2-2]
	\arrow["\lrcorner"{anchor=center, pos=0.125, rotate=180}, draw=none, from=2-2, to=1-1]
	\arrow[from=2-3, to=2-4]
\end{tikzcd}\]
are not inverse of each other, nor are they even equivalences.
A spectrum is a sequence $(X_i)_{i \geqslant 0}$ of pointed spaces together with equivalences $X_i \simeq \Omega X_{i+1}$ ; this construction effectively formally inverts the suspension and loop space functors and make them inverse equivalences of each other.

It turns out that spectra are much easier to define as an $\infty$-category than as a (model) 1-category :
\begin{definition}
    Let $\Eq(\Spc_*)$ be the full subcategory of $\Spc_*^{\Delta^1}$ on arrows which are equivalences.
    It bears two functors $s, t : \Eq(\Spc_*) \to \Spc_*$, namely the source and target functor.
    Define the $\infty$-category of \emph{spectra} $\Sp = \Eq(\Spc_*) \times_{\Spc_*} \Eq(\Spc_*) \times_{\Spc_*} \dots$ as the limit (infinite iterated pullback) in $\Catinf$ of the following diagram :
    % https://q.uiver.app/#q=WzAsNyxbMCwwLCJcXEVxKFxcU3BjXyopIl0sWzEsMSwiXFxTcGNfKiJdLFsyLDAsIlxcRXEoXFxTcGNfKikiXSxbMywxLCJcXFNwY18qIl0sWzQsMCwiXFxFcShcXFNwY18qKSJdLFs1LDEsIlxcZG90cyJdLFs2LDAsIlxcZG90cyJdLFswLDEsInQiXSxbMiwxLCJcXE9tZWdhIFxcY2lyYyBzIiwyXSxbMiwzLCJ0Il0sWzQsMywiXFxPbWVnYSBcXGNpcmMgcyIsMl0sWzQsNSwidCJdLFs2LDUsIlxcT21lZ2EgXFxjaXJjIHMiLDJdXQ==
    \[\begin{tikzcd}
        {\Eq(\Spc_*)} && {\Eq(\Spc_*)} && {\Eq(\Spc_*)} && \dots \\
        & {\Spc_*} && {\Spc_*} && \dots
        \arrow["t", from=1-1, to=2-2]
        \arrow["{\Omega \circ s}"', from=1-3, to=2-2]
        \arrow["t", from=1-3, to=2-4]
        \arrow["{\Omega \circ s}"', from=1-5, to=2-4]
        \arrow["t", from=1-5, to=2-6]
        \arrow["{\Omega \circ s}"', from=1-7, to=2-6]
    \end{tikzcd}\]
\end{definition}
In simpler words, objects of $\Sp$ are lists $(X_i)_{i \geq 0}$ of pointed spaces together with a list of equivalences $X_i \simeq \Omega X_{i+1}$, and maps $X \to Y$ are lists of pointed maps $f_i : X_i \to Y_i$ together with a list of chosen homotopies filling the following square :
% https://q.uiver.app/#q=WzAsNCxbMCwwLCJYX2kiXSxbMSwwLCJZX2kiXSxbMCwxLCJcXE9tZWdhIFhfe2krMX0iXSxbMSwxLCJcXE9tZWdhIFlfe2krMX0iXSxbMCwxLCJmX2kiXSxbMiwzLCJcXE9tZWdhIGZfe2krMX0iLDJdLFswLDIsIlxcc2ltZXEiLDJdLFsxLDMsIlxcc2ltZXEiXSxbMSwyLCIiLDEseyJzaG9ydGVuIjp7InNvdXJjZSI6MjAsInRhcmdldCI6MjB9LCJsZXZlbCI6Mn1dXQ==
\[\begin{tikzcd}
	{X_i} & {Y_i} \\
	{\Omega X_{i+1}} & {\Omega Y_{i+1}}
	\arrow["{f_i}", from=1-1, to=1-2]
	\arrow["\simeq"', from=1-1, to=2-1]
	\arrow[shorten <=4pt, shorten >=4pt, Rightarrow, from=1-2, to=2-1]
	\arrow["\simeq", from=1-2, to=2-2]
	\arrow["{\Omega f_{i+1}}"', from=2-1, to=2-2]
\end{tikzcd}\]

\begin{remark}
    When passing from 1-categories to $\infty$-categories, the category $\Set$ is replaced by the $\infty$-category $\Spc$.
    Following this idea, the category $\mathrm{Ab}$ of abelian groups has to be replaced by the $\infty$-category $\Sp_{\geqslant 0}$ of connective spectra (those whose negative homotopy groups vanish).
    Indeed, $\mathrm{Ab}$ is the category of commutative monoids in $\Set$ which are grouplike ; and by May's recognition theorem, the grouplike commutative monoids in $\Spc$ are exactly the connective spectra.
\end{remark}
\end{document}