\documentclass[11pt]{article}

\usepackage{quiver}
\usepackage{amsmath, amsthm, mathrsfs}
\usepackage[margin = 3cm]{geometry}
\usepackage{enumitem}
\usepackage{csquotes}

\newtheorem{theorem}{Theorem}[section]
\newtheorem{proposition}[theorem]{Proposition}
\newtheorem{lemma}[theorem]{Lemma}
\newtheorem{corollary}[theorem]{Corollary}
\theoremstyle{definition}
\newtheorem{definition}[theorem]{Definition}
\newtheorem{remark}[theorem]{Remark}
\newtheorem{example}[theorem]{Example}
\newtheorem{exercise}[theorem]{Exercise}

\newcommand{\A}{\mathbf{A}}
\newcommand{\Ani}{\operatorname{Ani}}
\newcommand{\ani}{\mathrm{ani}}
\newcommand{\C}{\mathcal{C}}
\newcommand{\CAlg}{\mathrm{CAlg}}
\newcommand{\Cart}{\operatorname{Cart}}
\newcommand{\Cat}{\mathrm{Cat}}
\newcommand{\Catinf}{\mathcal{C}\mathrm{at}_{\infty}}
\newcommand{\cc}{\mathrm{cc}}
\newcommand{\Cof}{\mathrm{Cof}}
\newcommand{\colim}{\operatorname*{colim}}
\newcommand{\Cut}{\operatorname{Cut}}
\newcommand{\D}{\mathcal{D}}
\newcommand{\E}{\mathcal{E}}
\newcommand{\Einf}{\mathbb{E}_{\infty}}
\newcommand{\Eq}{\operatorname{Eq}}
\newcommand{\Fib}{\mathrm{Fib}}
\newcommand{\Fun}{\operatorname{Fun}}
\newcommand{\Grpd}{\mathrm{Grpd}}
\newcommand{\h}{\mathrm{h}}
\newcommand{\heart}{\heartsuit}
\newcommand{\Hom}{\operatorname{Hom}}
\newcommand{\iHom}{\operatorname{\underline{Hom}}}
\newcommand{\Ind}{\operatorname{Ind}}
\newcommand{\J}{\mathcal{J}}
\newcommand{\Kan}{\mathrm{Kan}}
\renewcommand{\L}{\mathbb{L}}
\newcommand{\M}{\mathscr{M}}
\newcommand{\Map}{\operatorname{Map}}
\newcommand{\N}{\mathrm{N}}
\newcommand{\Ob}{\operatorname{Ob}}
\newcommand{\op}{\mathrm{op}}
\newcommand{\p}{\mathfrak{p}}
\renewcommand{\P}{\mathcal{P}}
\newcommand{\Poly}{\mathrm{Poly}}
\newcommand{\proj}{\mathrm{proj}}
\newcommand{\Q}{\mathbf{Q}}
\newcommand{\QCoh}{\operatorname{QCoh}}
\newcommand{\SCR}{\mathrm{SCR}}
\newcommand{\Set}{\mathrm{Set}}
\newcommand{\Sp}{\mathcal{S}\mathrm{p}}
\newcommand{\Spc}{\mathcal{S}\mathrm{pc}}
\newcommand{\Spec}{\operatorname{Spec}}
\newcommand{\sSet}{\mathrm{sSet}}
\newcommand{\Tor}{\operatorname{Tor}}
\newcommand{\X}{\mathcal{X}}
\newcommand{\Y}{\mathcal{Y}}
\newcommand{\Z}{\mathbf{Z}}


\title{Lecture notes in Derived Algebraic Geometry\\Session 3}
\date{17 February 2025}
\author{Course by F. Binda, notes by E. Hecky}


\begin{document}
\maketitle

\section{A word about simplicial rings and $\Einf$-rings}

Recall from the previous session that if $C$ is a category with coproducts, then $\Ani(C) = \P_{\Sigma}(C^{1-\omega\proj})$ is its animation.
In particular, $\CAlg^{\ani} = \Fun^{\Pi}(\Poly_{\Z}^{\op}, \Spc)$ is the $\infty$-category of animated rings, and it contains fully faithfully the usual category $\CAlg^{\heart}$ as the subcategory of $0$-truncated objects.

\begin{remark}
    Denoting by $\SCR$ the simplicial category of simplicial commutative rings, there is an equivalence
    \[
        \infty(\N_{\Delta}(\SCR)) \simeq \CAlg^{\ani}
    \]
    between animated rings and the $\infty$-category associated to the simplicial nerve of $\SCR$.
    That is the main reason why the simplicial approach of Toën-Vezzosi gives the same theory in the end as the more modern approach using animated rings.
\end{remark}

Another approach that is linked to animated rings is the theory of $\Einf$-rings.
Recall that the $\infty$-category $\Sp$ of spectra admits a structure of a symmetric monoidal stable $\infty$-category.
In particular, it is possible to consider commutative algebra objects, and we get the $\infty$-category $\CAlg(\Sp)$ of $\Einf$-rings spectra.

One can associate to any polynomial algebra a connective ring spectrum, yielding a functor
\[
    \Poly_{\Z} \to \CAlg(\Sp_{\geqslant 0})
\]
which then extends to
\[
    \CAlg^{\ani} \to \Einf\text{-rings}.
\]

However, this functor is not essentially surjective.
Indeed, if $A$ is an animated ring, then $\pi_*(A)$ is a graded commutative ring, each $\pi_n(A)$ is a module over the classical ring $\pi_0(A)$.
More importantly, the even homotopy groups $\pi_{2*}(A)$ have a structure of a divided power algebra\footnote{if $A$ is an algebra over the rational numbers $\Q$, then this structure is just the usual division by integers.}.

This is not the case for every $\Einf$-ring spectrum.
For example, the algebra $\mathbb{F}_p[X]$ with $X$ of degree $2$ does not admit a PD-structure, so it doesn't come from any animated ring.

That being said, this phenomenon purely comes from torsion, and if we consider rational algebras, we get an equivalence :
\[
    \CAlg_{\Q}^{\ani} \simeq \CAlg(\Sp_{\geqslant 0})_{\Q/}.
\]

\section{Base change}

\begin{lemma}
    Let $\varphi : R \to R'$ be a morphism in $\CAlg^{\heart}$.
    It induces a functor $\Poly_R \to \Poly_{R'}$ sending $P$ to $P \otimes_R R'$, inducing an adjunction :
    \[
        \CAlg_R^{\ani} \rightleftarrows \CAlg_{R'}^{\ani}
    \]
    between extension and restriction of scalars.
\end{lemma}
\begin{proof}
    In the following diagram, the diagonal composition descends to a unique extension of scalars functor :
    % https://q.uiver.app/#q=WzAsNCxbMCwwLCJcXFBvbHlfUiJdLFsxLDAsIlxcUG9seV97Uid9Il0sWzEsMSwiXFxDQWxnX3tSJ31ee1xcYW5pfSJdLFswLDEsIlxcQ0FsZ19SXntcXGFuaX0iXSxbMCwxLCItIFxcb3RpbWVzX1IgUiciXSxbMCwzLCJqIiwyLHsic3R5bGUiOnsidGFpbCI6eyJuYW1lIjoiaG9vayIsInNpZGUiOiJ0b3AifX19XSxbMSwyLCJqIiwwLHsic3R5bGUiOnsidGFpbCI6eyJuYW1lIjoiaG9vayIsInNpZGUiOiJ0b3AifX19XSxbMCwyXSxbMywyLCJcXGV4aXN0cyEiLDIseyJzdHlsZSI6eyJib2R5Ijp7Im5hbWUiOiJkYXNoZWQifX19XV0=
    \[\begin{tikzcd}
        {\Poly_R} & {\Poly_{R'}} \\
        {\CAlg_R^{\ani}} & {\CAlg_{R'}^{\ani}}
        \arrow["{- \otimes_R R'}", from=1-1, to=1-2]
        \arrow["j"', hook, from=1-1, to=2-1]
        \arrow[from=1-1, to=2-2]
        \arrow["j", hook, from=1-2, to=2-2]
        \arrow["{\exists!}"', dashed, from=2-1, to=2-2]
    \end{tikzcd}\]
    as the image of $j \circ (- \otimes_R R')$ by the equivalence
    \[
        \Fun_{\Sigma}(\Poly_R, \CAlg_{R'}^{\ani}) \simeq \Fun'(\CAlg_R^{\ani}, \CAlg_{R'}^{\ani}).
    \]
    The existence of a right adjoint comes from presentability and the adjoint functor theorem.
\end{proof}

This construction yields the same result as the \enquote{naive} idea of taking $R \in \CAlg^{\heart}$ and $S, T \in \CAlg_R^{\ani}$, and defining $S \otimes_R^{\L} T$ as the corresponding pushout of animated algebras.
To see this, just notice that they have the exact same behavior on polynomial algebras, and both constructions preserve enough colimits to extend to the same functor.

\begin{remark}
    There is an equivalence $\Ani(\CAlg_R^{\heart}) \simeq \CAlg^{\ani}_{R/}$ between the animation of the category of (classical) $R$-algebras and the slice category of animated algebras under $R$.
\end{remark}

This remark being true for static rings $R$ motivates the following definition for all animated rings.

\begin{definition}
    Let $A \in \CAlg^{\ani}$ be an animated ring.
    Then define the $\infty$-category of animated $A$-algebras as :
    \[
        \CAlg_A^{\ani} = \Ani(\CAlg^{\heart})_{A/}.
    \]
\end{definition}
In fact this is even functorial in $A$, in the sense that a functor
\[
    \CAlg_{(-)}^{\ani} : \Ani(\CAlg^{\heart}) \to \Catinf
\]
can be obtained by straightening the extension/restriction of scalars adjunction constructed above.
\end{document}