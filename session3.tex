\documentclass[11pt]{article}

\usepackage{quiver}
\usepackage{amsmath, amsthm, mathrsfs}
\usepackage[margin = 3cm]{geometry}
\usepackage{enumitem}
\usepackage{csquotes}
\usepackage{stmaryrd}

\newtheorem{theorem}{Theorem}[section]
\newtheorem{proposition}[theorem]{Proposition}
\newtheorem{lemma}[theorem]{Lemma}
\newtheorem{corollary}[theorem]{Corollary}
\theoremstyle{definition}
\newtheorem{definition}[theorem]{Definition}
\newtheorem{remark}[theorem]{Remark}
\newtheorem{example}[theorem]{Example}
\newtheorem{exercise}[theorem]{Exercise}

\newcommand{\A}{\mathbf{A}}
\newcommand{\Ani}{\operatorname{Ani}}
\newcommand{\ani}{\mathrm{ani}}
\newcommand{\C}{\mathcal{C}}
\newcommand{\CAlg}{\mathrm{CAlg}}
\newcommand{\Cart}{\operatorname{Cart}}
\newcommand{\Cat}{\mathrm{Cat}}
\newcommand{\Catinf}{\mathcal{C}\mathrm{at}_{\infty}}
\newcommand{\cc}{\mathrm{cc}}
\newcommand{\Cof}{\mathrm{Cof}}
\newcommand{\colim}{\operatorname*{colim}}
\newcommand{\CRMod}{\mathrm{CRMod}}
\newcommand{\Cut}{\operatorname{Cut}}
\newcommand{\D}{\mathcal{D}}
\newcommand{\Der}{\operatorname{Der}}
\newcommand{\E}{\mathcal{E}}
\newcommand{\Einf}{\mathbb{E}_{\infty}}
\newcommand{\Eq}{\operatorname{Eq}}
\newcommand{\et}{\mathrm{\acute{e}t}}
\newcommand{\FFree}{\mathrm{FFree}}
\newcommand{\Fib}{\mathrm{Fib}}
\newcommand{\Fun}{\operatorname{Fun}}
\newcommand{\Grpd}{\mathrm{Grpd}}
\newcommand{\h}{\mathrm{h}}
\newcommand{\heart}{\heartsuit}
\newcommand{\Hom}{\operatorname{Hom}}
\newcommand{\iHom}{\operatorname{\underline{Hom}}}
\newcommand{\Ind}{\operatorname{Ind}}
\newcommand{\J}{\mathcal{J}}
\newcommand{\Kan}{\mathrm{Kan}}
\renewcommand{\L}{\mathbb{L}}
\newcommand{\M}{\mathscr{M}}
\newcommand{\Map}{\operatorname{Map}}
\newcommand{\Mod}{\mathrm{Mod}}
\newcommand{\N}{\mathrm{N}}
\newcommand{\Ob}{\operatorname{Ob}}
\newcommand{\op}{\mathrm{op}}
\newcommand{\p}{\mathfrak{p}}
\renewcommand{\P}{\mathcal{P}}
\newcommand{\Poly}{\mathrm{Poly}}
\newcommand{\PrL}{\mathrm{Pr}^{\mathrm{L}}}
\newcommand{\PrR}{\mathrm{Pr}^{\mathrm{R}}}
\newcommand{\pr}{\mathrm{pr}}
\newcommand{\proj}{\mathrm{proj}}
\newcommand{\Q}{\mathbf{Q}}
\newcommand{\QCoh}{\operatorname{QCoh}}
\newcommand{\SCR}{\mathrm{SCR}}
\newcommand{\Set}{\mathrm{Set}}
\newcommand{\Sp}{\mathcal{S}\mathrm{p}}
\newcommand{\Spc}{\mathcal{S}\mathrm{pc}}
\newcommand{\Spec}{\operatorname{Spec}}
\newcommand{\sSet}{\mathrm{sSet}}
\newcommand{\Tor}{\operatorname{Tor}}
\newcommand{\X}{\mathcal{X}}
\newcommand{\Y}{\mathcal{Y}}
\newcommand{\Z}{\mathbf{Z}}


\title{Lecture notes in Derived Algebraic Geometry\\Session 3}
\date{17 February 2025}
\author{Course by F. Binda, notes by E. Hecky}


\begin{document}
\maketitle

\section{A word about simplicial rings and $\Einf$-rings}

Recall from the previous session that if $C$ is a category with coproducts, then $\Ani(C) = \P_{\Sigma}(C^{1-\omega\proj})$ is its animation.
In particular, $\CAlg^{\ani} = \Fun^{\Pi}(\Poly_{\Z}^{\op}, \Spc)$ is the $\infty$-category of animated rings, and it contains fully faithfully the usual category $\CAlg^{\heart}$ as the subcategory of $0$-truncated objects.

\begin{remark}
    Denoting by $\SCR$ the simplicial category of simplicial commutative rings, there is an equivalence
    \[
        \infty(\N_{\Delta}(\SCR)) \simeq \CAlg^{\ani}
    \]
    between animated rings and the $\infty$-category associated to the simplicial nerve of $\SCR$.
    That is the main reason why the simplicial approach of Toën-Vezzosi gives the same theory in the end as the more modern approach using animated rings.
\end{remark}

Another approach that is linked to animated rings is the theory of $\Einf$-rings.
Recall that the $\infty$-category $\Sp$ of spectra admits a structure of a symmetric monoidal stable $\infty$-category.
In particular, it is possible to consider commutative algebra objects, and we get the $\infty$-category $\CAlg(\Sp)$ of $\Einf$-rings spectra.

One can associate to any polynomial algebra a connective ring spectrum, yielding a functor
\[
    \Poly_{\Z} \to \CAlg(\Sp_{\geqslant 0})
\]
which then extends to
\[
    \CAlg^{\ani} \to \Einf\text{-rings}.
\]

However, this functor is not essentially surjective.
Indeed, if $A$ is an animated ring, then $\pi_*(A)$ is a graded commutative ring, each $\pi_n(A)$ is a module over the classical ring $\pi_0(A)$.
More importantly, the even homotopy groups $\pi_{2*}(A)$ have a structure of a divided power algebra\footnote{if $A$ is an algebra over the rational numbers $\Q$, then this structure is just the usual division by integers.}.

This is not the case for every $\Einf$-ring spectrum.
For example, the algebra $\mathbb{F}_p[X]$ with $X$ of degree $2$ does not admit a PD-structure, so it doesn't come from any animated ring.

That being said, this phenomenon purely comes from torsion, and if we consider rational algebras, we get an equivalence :
\[
    \CAlg_{\Q}^{\ani} \simeq \CAlg(\Sp_{\geqslant 0})_{\Q/}.
\]

\section{Base change}

\begin{lemma}
    Let $\varphi : R \to R'$ be a morphism in $\CAlg^{\heart}$.
    It induces a functor $\Poly_R \to \Poly_{R'}$ sending $P$ to $P \otimes_R R'$, inducing an adjunction :
    \[
        \CAlg_R^{\ani} \rightleftarrows \CAlg_{R'}^{\ani}
    \]
    between extension and restriction of scalars.
\end{lemma}
\begin{proof}
    In the following diagram, the diagonal composition descends to a unique extension of scalars functor :
    % https://q.uiver.app/#q=WzAsNCxbMCwwLCJcXFBvbHlfUiJdLFsxLDAsIlxcUG9seV97Uid9Il0sWzEsMSwiXFxDQWxnX3tSJ31ee1xcYW5pfSJdLFswLDEsIlxcQ0FsZ19SXntcXGFuaX0iXSxbMCwxLCItIFxcb3RpbWVzX1IgUiciXSxbMCwzLCJqIiwyLHsic3R5bGUiOnsidGFpbCI6eyJuYW1lIjoiaG9vayIsInNpZGUiOiJ0b3AifX19XSxbMSwyLCJqIiwwLHsic3R5bGUiOnsidGFpbCI6eyJuYW1lIjoiaG9vayIsInNpZGUiOiJ0b3AifX19XSxbMCwyXSxbMywyLCJcXGV4aXN0cyEiLDIseyJzdHlsZSI6eyJib2R5Ijp7Im5hbWUiOiJkYXNoZWQifX19XV0=
    \[\begin{tikzcd}
        {\Poly_R} & {\Poly_{R'}} \\
        {\CAlg_R^{\ani}} & {\CAlg_{R'}^{\ani}}
        \arrow["{- \otimes_R R'}", from=1-1, to=1-2]
        \arrow["j"', hook, from=1-1, to=2-1]
        \arrow[from=1-1, to=2-2]
        \arrow["j", hook, from=1-2, to=2-2]
        \arrow["{\exists!}"', dashed, from=2-1, to=2-2]
    \end{tikzcd}\]
    as the image of $j \circ (- \otimes_R R')$ by the equivalence
    \[
        \Fun_{\Sigma}(\Poly_R, \CAlg_{R'}^{\ani}) \simeq \Fun'(\CAlg_R^{\ani}, \CAlg_{R'}^{\ani}).
    \]
    The existence of a right adjoint comes from presentability and the adjoint functor theorem.
\end{proof}

This construction yields the same result as the \enquote{naive} idea of taking $R \in \CAlg^{\heart}$ and $S, T \in \CAlg_R^{\ani}$, and defining $S \otimes_R^{\L} T$ as the corresponding pushout of animated algebras.
To see this, just notice that they have the exact same behavior on polynomial algebras, and both constructions preserve enough colimits to extend to the same functor.

\begin{remark}
    There is an equivalence $\Ani(\CAlg_R^{\heart}) \simeq \CAlg^{\ani}_{R/}$ between the animation of the category of (classical) $R$-algebras and the slice category of animated algebras under $R$.
\end{remark}

This remark being true for static rings $R$ motivates the following definition for all animated rings.

\begin{definition}
    Let $A \in \CAlg^{\ani}$ be an animated ring.
    Then define the $\infty$-category of animated $A$-algebras as :
    \[
        \CAlg_A^{\ani} = \Ani(\CAlg^{\heart})_{A/}.
    \]
\end{definition}
In fact this is even functorial in $A$, in the sense that a functor
\[
    \CAlg_{(-)}^{\ani} : \Ani(\CAlg^{\heart}) \to \Catinf
\]
can be obtained by straightening the extension/restriction of scalars adjunction constructed above.

\section{Animated modules}

Let $R \in \CAlg^{\heart}$ be a classical ring.
Then the compact projective $R$-modules $(\Mod_R^{\heart})^{\omega\proj}$ are the retracts of finite free modules.
One then gets the $\infty$-category of animated $R$-modules by taking $\Sigma$-presheaves of spaces on those modules.

It is possible to promote this construction in two different ways :
\begin{itemize}
    \item allow $R$ to be an animated ring ;
    \item make this construction functorial in $R$.
\end{itemize}
This section's goal is to do both at once.

Let $\CRMod^{\heart}$ be the following 1-category.
Its objects are pairs $(A, M)$ with $A \in \CAlg^{\heart}$ a ring and $M \in \Mod_A^{\heart}$ an $A$-module.
A morphism $(A, M) \to (B, N)$ consists of a ring map $f : A \to B$, and of a $B$-linear map $\varphi : M \otimes_A B \to N$ (or equivalently by adjunction, an $A$-linear map from $M$ to the restriction of $N$).

\begin{proposition}
    The compact projective objects of $\CRMod^{\heart}$ are the (retracts of) finite free modules over (retracts of) finite polynomial algebras over $\Z$.
    In other words, up to retracts, every compact projective object in this category looks like $(\Z[\underline{X}], \Z[\underline{X}]^{\oplus n})$ with $\underline{X}$ a finite set of unknowns and $n \geqslant 0$.
\end{proposition}
\begin{proof}[Ideas of the proof]
    Let $C$ be the category which is claimed to be $(\CRMod^{\heart})^{\omega\proj}$ in the statement.
    Notice that every object in $\CRMod^{\heart}$ is a 1-sifted colimit of objects of $C$.
    Also, $C$ is the closure under finite coproducts of the category spanned by the two generating objects $(\Z[T], 0)$ and $(\Z, \Z)$.
    Finally, both these objects are themselves compact projective.
\end{proof}

Keeping this notation, the animation $\Ani(\CRMod^{\heart})$ is the $\infty$-category of presheaves of spaces on this $C$.
We then get two projection functors :
\[
    \CAlg^{\ani} \xleftarrow{\pr_1} \Ani(\CRMod^{\heart}) \xrightarrow{\pr_2} \Mod_{\Z}^{\ani}.
\]
Notice that both functors have sections.

\begin{remark}
    For every static ring $R \in \CAlg^{\heart}$, the forgetful functor
    \[
        \Ani(\CRMod^{\heart}_R) \to \CAlg_R^{\ani} \times \Mod_R^{\ani}
    \]
    is conservative.
\end{remark}

This $\infty$-category together with the inner fibration $\pr_1$ gives the desired definition and functoriality of the $\infty$-categories of animated modules.

\begin{definition}
    For every animated ring $A \in \CAlg^{\ani}$, let $\Mod_A$ be the fiber of $\pr_1$ over $A$.
    In other words, it is the following pullback
    % https://q.uiver.app/#q=WzAsNCxbMCwwLCJcXE1vZF9BIl0sWzEsMCwiXFxBbmkoXFxDUk1vZF57XFxoZWFydH0pIl0sWzEsMSwiXFxDQWxnXntcXGFuaX0iXSxbMCwxLCJcXERlbHRhXjAiXSxbMywyLCJBIiwyXSxbMSwyLCJcXHByXzEiXSxbMCwzXSxbMCwxXSxbMCwyLCIiLDEseyJzdHlsZSI6eyJuYW1lIjoiY29ybmVyIn19XV0=
    \[\begin{tikzcd}
        {\Mod_A} & {\Ani(\CRMod^{\heart})} \\
        {\Delta^0} & {\CAlg^{\ani}}
        \arrow[from=1-1, to=1-2]
        \arrow[from=1-1, to=2-1]
        \arrow["\lrcorner"{anchor=center, pos=0.125}, draw=none, from=1-1, to=2-2]
        \arrow["{\pr_1}", from=1-2, to=2-2]
        \arrow["A"', from=2-1, to=2-2]
    \end{tikzcd}\]
    and it is indeed an $\infty$-category because $\pr_1$ is an inner fibration.
\end{definition}

Actually, $\pr_1$ is fortunately more than an inner fibration.
It is both a Cartesian and a coCartesian fibration, whence a functoriality of $\Mod_(-)$ both covariantly and contravariantly.
\begin{theorem}
    $\pr_1$ is a biCartesian fibration.
\end{theorem}
\begin{proof}
    Let us prove that it is coCartesian.
    Concretely, for every animated ring map $f : A \to B$, we need to find $\mathcal{A}, \mathcal{B} \in \Ani(\CRMod^{\heart})$ whose $\pr_1$ are respectively $A$ and $B$, and a map $\phi : \mathcal{A} \to \mathcal{B}$ such that for every $X \in \Ani(\CRMod^{\heart})$, the following diagram is homotopy-cartesian in $\Spc$ :
    % https://q.uiver.app/#q=WzAsNCxbMCwwLCJcXE1hcChcXG1hdGhjYWx7Qn0sIFgpIl0sWzEsMCwiXFxNYXAoXFxtYXRoY2Fse0F9LCBYKSJdLFswLDEsIlxcTWFwKEIsIFxccHJfMShYKSkiXSxbMSwxLCJcXE1hcChBLCBcXHByXzEoWCkpIl0sWzAsMSwiXFxwaGkiXSxbMSwzLCJcXHByXzEiXSxbMCwyLCJcXHByXzEiLDJdLFsyLDMsImYiLDJdXQ==
    \[\begin{tikzcd}
        {\Map(\mathcal{B}, X)} & {\Map(\mathcal{A}, X)} \\
        {\Map(B, \pr_1(X))} & {\Map(A, \pr_1(X))}
        \arrow["\phi", from=1-1, to=1-2]
        \arrow["{\pr_1}"', from=1-1, to=2-1]
        \arrow["{\pr_1}", from=1-2, to=2-2]
        \arrow["f"', from=2-1, to=2-2]
    \end{tikzcd}\]
    The homotopy-cartesianity condition is equivalent to the fibers of the vertical arrows being homotopy equivalent.

    Since every possible choice of $\mathcal{A}$ and $\mathcal{B}$ are sifted colimits of compact projective objects, and since pullbacks commute with cosifted limits, it will suffice to treat the case of $A$ and $B$ compact projective, and to look for $\mathcal{A}$ and $\mathcal{B}$ compact projective as well.
    By definition of compact projectivity, it also suffices to treat the case where $X$ itself is compact projective.

    Now the mapping spaces in the diagram have all become discrete sets !
    Write $\mathcal{A} = (A, M)$, $\mathcal{B} = (B, N)$ and $X = (\pr_1(X), M_X)$.
    Take $\psi : B \to \pr_1(X)$ and compare $\pr_1^{-1}(\psi) \in \Map(\mathcal{B}, X)$ which we can describe as
    \[
        \pr_1^{-1}(\psi) = \{\tilde{\psi} : (B, N) \to (\pr_1(X), M_X) \mid \pr_1(\tilde{\psi}) = \psi\} \simeq \Hom_{\Mod_{\pr_1(X)}}(N \otimes_B \pr_1(X), M_X),
    \]
    with $\pr_1^{-1}(\psi \circ f) \in \Map(\mathcal{A}, X)$ which can similarly be described as
    \[
        \pr_1^{-1}(\psi \circ f) \simeq \Hom_{\Mod_{\pr_1(X)}}((M \otimes_A B) \otimes_B \pr_1(X), M_X).
    \]
    
    Of course, only $f : A \to B$ is given, and we are free to choose $M$ and $N$.
    We just need a morphism $\pr_2(\phi) : M \otimes_A B \to N$ such that $\pr_2(\phi) \otimes \pr_1(X)$ is an equivalence for every choice of $X$.

    One can take $M = A$ and $N = B$ (and the fibers are both $M_X$), or even more stupidly, one can take $M = 0$ and $N = 0$ (and the fibers are both singletons).
\end{proof}

This is the exact theorem that was needed to show the functoriality of the $\infty$-category of animated modules.
We get two functors :
\begin{itemize}
    \item $\Mod \in \Fun(\CAlg^{\ani, \op}, \PrR)$ sending $\phi : A \to A'$ to the $\infty$-functor $\Mod(\phi) : \Mod_{A'} \to \Mod_A$ given by (generalized) restriction of scalars ;
    \item $\Mod^* \in \Fun(\CAlg^{\ani}, \mathrm{Pr}^{\mathrm{L}, \otimes})$ sending $A \to A'$ to the symmetric monoidal $\infty$-functor of (generalized) extension of scalars $- \otimes_A^{\L} A' : \Mod_A \to \Mod_{A'}$.
\end{itemize}

\begin{remark}
    There is another possible construction for animated modules, which is to define finite free animated $A$-modules for $A \in \CAlg^{\ani}$ as $\FFree_A = A \otimes_{\Z}^{\L} \FFree_{\Z}$
    and then animate this to get $\Mod_A$.
\end{remark}

\section{Geometrical notions animated}

This section defines flat and étale maps of animated modules, as well as how one can perform animated localizations and quotients.

\subsection{Flatness}

\begin{definition}
    Let $A \in \CAlg^{\ani}$ be an animated ring and $M \in \Mod_A$ be a (connective) $A$-module.
    $M$ is \emph{flat} if one/all of the following equivalent conditions holds :
    \begin{enumerate}
        \item (Homological flatness) The functor $M \otimes^{\L} -$ is left exact (i.e. it preserves finite limits) ;
        \item (No higher Tor) For every $N \in \Mod_A$ which is $n$-truncated, the tensor product $M \otimes^{\L} N$ is also $n$-truncated ;
        \item (Fiber-flatness) The following conditions hold :
        \begin{enumerate}
            \item $\pi_0(M)$ is a flat $\pi_0(A)$-module,
            \item for every $i \geq 0$, the map
            \[
                \pi_i(A) \otimes_{\pi_0(A)} \pi_0(M) \to \pi_i(M)
            \]
            is an equivalence.
        \end{enumerate}
    \end{enumerate}
\end{definition}

\begin{definition}
    An animated ring map $f : A \to B$ is called \emph{flat} if $B$ is flat as an $A$-module.
    It is called \emph{faithfully flat} if it is flat and $- \otimes_A^{\L} B$ is a conservative functor.
\end{definition}

\begin{theorem}[Lurie, classically Lazard]
    Let $A$ be an animated ring and $M \in \Mod_A$.
    Then $M$ is flat if and only if it is a filtered colimit of finite free $A$-modules.
\end{theorem}

\subsection{Localization}

When localizing, one has to be careful : trying to invert an element in a positive homotopy group of a ring forces us to create elements in the negative homotopy groups, infinitely far down - so localizing with respect to a positive homotopy class completely destroys any connectivity.
This subsection is only about how to localize with respect to an element in the $\pi_0$ of an animated ring.

Let $A$ be an animated ring.
Its underlying space $\Map(\Z[T], A)$ contains the underlying space of units $\Map(\Z[T^{\pm 1}], A)$, which we awkwardly denote by $A^{\times}$ here.
The inclusion induces an identification $\pi_0(A^{\times}) = \pi_0(A)^{\times}$.

\begin{proposition}
    Let $f : A \to B$ be an animated ring map, and $a \in \pi_0(A)$ be such that $f(a) \in \pi_0(B)^{\times}$.
    The following are then equivalent :
    \begin{enumerate}
        \item For every animated ring $R$, we have $\Map(B, R) \simeq \Map'(A, R)$ where $\Map'$ denotes the maps which send $a$ to an invertible element in $R$ ;
        \item For every $n$, there is an equivalence
        \[
            \pi_n(B) \simeq \pi_n(A) \otimes_{\pi_0(A)} \pi_0(A)[1/a].
        \]
    \end{enumerate}
\end{proposition}
When this is the case, we say that $B$ is a/the \emph{localization of $A$ with respect to $a$}.
The module $B$ is then automatically flat over $A$, as in classical algebraic geometry.

How can one construct such a localization ?
It is not hard, one can follow the same process as in classical ring theory : just construct the localization as the derived tensor product
% https://q.uiver.app/#q=WzAsNCxbMCwwLCJcXFpbVF0iXSxbMSwwLCJBIl0sWzAsMSwiXFxaW1Ree1xccG0gMX1dIl0sWzEsMSwiQiA9IEEgXFxvdGltZXNfe1xcWltUXX1ee1xcTH0gXFxaW1Ree1xccG0gMX1dIl0sWzAsMSwiYSJdLFswLDJdLFsyLDNdLFsxLDNdLFszLDAsIiIsMSx7InN0eWxlIjp7Im5hbWUiOiJjb3JuZXIifX1dXQ==
\[\begin{tikzcd}
	{\Z[T]} & A \\
	{\Z[T^{\pm 1}]} & {B = A \otimes_{\Z[T]}^{\L} \Z[T^{\pm 1}]}
	\arrow["a", from=1-1, to=1-2]
	\arrow[from=1-1, to=2-1]
	\arrow[from=1-2, to=2-2]
	\arrow[from=2-1, to=2-2]
	\arrow["\lrcorner"{anchor=center, pos=0.125, rotate=180}, draw=none, from=2-2, to=1-1]
\end{tikzcd}\]
Defining it this way automatically proves the proposition.
All choices of $B$, coming from all the possible choices of $a : \Z[T] \to A$, are contractibly equivalent.

With the exact same construction, it is of course possible to localize modules.

\subsection{Quotients}
Let $A$ be an animated ring and $f_1, \dots, f_n \in \pi_0(A)$ be elements, which we gather as a map $\Z[T_1, \dots, T_n] \to A$.
Then one can define the quotient, following the same strategy as the localization :
\[
    A\sslash(f_1, \dots, f_n) = A \otimes_{Z[T_1, \dots, T_n]}^{\L} \frac{\Z[T_1, \dots, T_n]}{(T_1, \dots, T_n)}.
\]

\begin{proposition}
    The quotient behaves like the usual quotient in classical ring theory :
    \begin{enumerate}
        \item $\pi_0(A\sslash(f_1, \dots, f_n)) \simeq \pi_0(A)/(f_1, \dots, f_n)$ ;
        \item If $A$ is a static local ring then $A\sslash(f_1, \dots, f_n)$ is equivalent to $A/(f_1, \dots, f_n)$ \emph{if and only if} $(f_1, \dots, f_n)$ is a regular sequence\footnote{this was to be expected, since we only want derived intersections of classical things to stay classical if they behaved well in the first place.} ;
        \item $A\sslash(f)$ is the cofiber of $f : A \to A$ (it was not clear that this cofiber had a ring structure), and more generally $A\sslash(f_1, \dots, f_n)$ is the tensor product of the cofibers of the $f_i : A \to A$\footnote{this is to be compared with the Koszul complex}.
    \end{enumerate}
\end{proposition}

\begin{exercise}
    A particular element of $\pi_0(A)$ is $0$.
    What is the quotient $A\sslash(0)$ ?
\end{exercise}

\begin{definition}
    A map of animated rings $f : A \to B$ is \emph{(formally) étale} if the following conditions hold :
    \begin{enumerate}
        \item $\pi_0(f) : \pi_0(A) \to \pi_0(B)$ is (formally) étale ;
        \item For every $n \geq 0$, the map
        \[
            \pi_n(A) \otimes_{\pi_0(A)} \pi_0(B) \to \pi_n(B)
        \]
        is an equivalence.
    \end{enumerate}
\end{definition}

\begin{theorem}[Lurie]
    Let $A$ be an animated ring.
    Then the functor
    \[
        \pi_0 : \CAlg_A^{\et} \to \left(\CAlg_{\pi_0(A)}^{\heart}\right)^{\et}
    \]
    is an equivalence.
\end{theorem}

\begin{theorem}[Lurie/Matthew/Toën-Vezzosi]
    For every faithfully flat animated ring map $A \to B$, the following diagram is a limit diagram in $\Catinf$ :
    % https://q.uiver.app/#q=WzAsNCxbMSwwLCJcXE1vZF9CIl0sWzIsMCwiXFxNb2Rfe0JcXG90aW1lcyBCfSJdLFszLDAsIlxcZG90cyJdLFswLDAsIlxcTW9kX0EiXSxbMywwXSxbMCwxLCIiLDAseyJvZmZzZXQiOi0xfV0sWzAsMSwiIiwwLHsib2Zmc2V0IjoxfV0sWzEsMl0sWzEsMiwiIiwwLHsib2Zmc2V0IjotMn1dLFsxLDIsIiIsMCx7Im9mZnNldCI6Mn1dXQ==
    \[\begin{tikzcd}
        {\Mod_A} & {\Mod_B} & {\Mod_{B\otimes B}} & \dots
        \arrow[from=1-1, to=1-2]
        \arrow[shift left, from=1-2, to=1-3]
        \arrow[shift right, from=1-2, to=1-3]
        \arrow[from=1-3, to=1-4]
        \arrow[shift left=2, from=1-3, to=1-4]
        \arrow[shift right=2, from=1-3, to=1-4]
    \end{tikzcd}\]
\end{theorem}
This theorem is the animated version of Grothendieck's faithfully flat descent.
In fact, the functor $\Mod^* \in \Fun(\CAlg^{\ani}, \PrL)$ constructed above is a sheaf for the flat topology.

\section{Square-zero extensions and the cotangent complex}

This last section aims to go towards the construction of the cotangent complex, which will be finished in the next session.

Recall that if $A \in \CAlg^{\heart}$ is a static commutative ring and $M \in \Mod_A^{\heart}$ is a static $A$-module, then one can construct $A \oplus M \in \CAlg_{A//A}^{\heart}$ as an augmented $A$-algebra.
The product in this algebra is given by the following formula :
\[
    (a, m) \cdot (a', m') = (aa', am' + a'm),
\]
which is a square-zero multiplication.

\begin{remark}
    By design, this augmented algebra corepresents ($\Z$-linear) derivations :
    \[
        \Der(A, M) \simeq \Hom_{\CAlg_{/A}^{\heart}}(A, A \oplus M).
    \]
    It sends a derivation $D$ to the map $s : a \mapsto (a, D(a))$.
\end{remark}
If we want to introduce derivations in an $\infty$-categorical context, there is a problem : how can one define the Leibniz rule $D(ab) = D(a)b + aD(b)$ without strict equality ?
The answer is that thanks to $A \oplus M$, we don't care : it suffices to define $A \oplus M$ and to define derivations as the things corepresented by this augmented algebra.

It looks like we have just moved the problem, because we now have to define the correct multiplication on $A \oplus M$.
There is a brutal model-categorical way to do this, but we don't have to.

It suffices to animate the functor $\CRMod^{\heart} \to \CAlg$ sending the static $(A, M)$ to $A \oplus M$ as defined above.
We get a functor $F : \Ani(\CRMod^{\heart}) \to \CAlg^{\ani}$, satisfying the following basic properties :
\begin{enumerate}
    \item $F$ preserves static objects ;
    \item If $M \simeq 0$ in $\Mod_A$, then $F(M) \simeq A$ in $\CAlg^{\ani}$ ;
    \item $A \oplus M \in \CAlg_{A//A}^{\ani}$ is an augmented $A$-algebra.
\end{enumerate}

The third point is really easy to understand once we have the second one : consider the stupid sequence $0 \to M \to 0$ in $\Mod_A$, and simply apply $F$.
Since $F(0) \simeq A$, we obtain the desired sequence $A \to A \oplus M \to A$.
\end{document}